\documentclass[12pt]{report}

\usepackage[english]{babel}
\usepackage[utf8x]{inputenc}
\usepackage{amsmath}
\usepackage{graphicx}
\usepackage[colorinlistoftodos]{todonotes}
\usepackage{hyperref}
\usepackage{multirow}
\usepackage{array,booktabs}
\usepackage{tikz}
\usetikzlibrary{arrows}
\usepackage{csquotes}
\usepackage{amssymb}
\usepackage{epstopdf}
\usepackage{wrapfig}
\usepackage{mathtools}

\DeclarePairedDelimiter\ceil{\lceil}{\rceil}
\DeclarePairedDelimiter\floor{\lfloor}{\rfloor}
\usepackage{pdfpages}
\usepackage{subcaption}

\newcommand{\todoin}{\todo[inline]}
\newcommand{\realR}{\mathbb{R}}
\newcommand{\real}[1]{\realR^{#1}}
\newcommand{\inreal}[1]{\in \real{#1}}

\newcommand{\weights}{w}
\newcommand{\w}{\weights}
\newcommand{\wk}[1]{\weights^{(#1)}}
\newcommand{\wki}[2]{\wk{#1}_{#2}}

\newcommand{\all}[1]{\expandafter\MakeUppercase\expandafter{#1}}
\newcommand{\num}[1]{\expandafter\MakeUppercase\expandafter{#1}}

\newcommand{\layer}{l}
\newcommand{\lk}[1]{\layer^{(#1)}}
\newcommand{\lki}[2]{\layer^{(#1)}_{#2}}

\newcommand{\numnodes}{m}
\newcommand{\m}{\numnodes}
\newcommand{\mk}[1]{\numnodes^{(#1)}}
\newcommand{\mki}[2]{\numnodes^{(#1)}_{#2}}

\newcommand{\outputvar}{o}
\newcommand{\ok}[1]{\outputvar^{(#1)}}
\newcommand{\oki}[2]{\outputvar^{(#1)}_{#2}}
\newcommand{\okT}[1]{\outputvar^{(#1)T}}
\newcommand{\okiT}[2]{\outputvar^{(#1)T}_{#2}}


\newcommand{\biasterm}{b}
\newcommand{\bk}[1]{\biasterm^{(#1)}}
\newcommand{\bki}[2]{\biasterm^{(#1)}_{#2}}

\newcommand{\layerfunction}{\psi}

\newcommand{\lf}{\layerfunction}
\newcommand{\lfp}{\lf'}
\newcommand{\lft}[1]{\lf^{(#1)}}
\newcommand{\lfk}[1]{\lf_{(#1)}}
\newcommand{\lftk}[2]{\lf^{(#1)}_{(#2)}}
\newcommand{\lfkt}[2]{\lf_{(#1)}^{(#2)}}
\newcommand{\lfpk}[1]{\lf'_{(#1)}}
\newcommand{\lfpkt}[2]{\lf_{(#1)}'^{(#2)}}

\newcommand{\FC}{\lft{FC}}
\newcommand{\FCk}[1]{\FC_{(#1)}}

\newcommand{\lftp}[1]{\lf'^{(#1)}}
\newcommand{\FCp}{\lftp{FC}}
\newcommand{\FCkp}[1]{\FCp_{(#1)}}

\newcommand{\conv}{\lft{Conv}}
\newcommand{\convk}[1]{\conv_{(#1)}}

\newcommand{\convp}{\lftp{Conv}}
\newcommand{\convkp}[1]{\convp_{(#1)}}

\newcommand{\maxpool}{\lft{maxpool}}
\newcommand{\maxpoolk}[1]{\maxpool_{(#1)}}

\newcommand{\avgpool}{\lft{avgpool}}
\newcommand{\avgpoolk}[1]{\avgpool_{(#1)}}

\newcommand{\activationfunction}{\sigma}
\newcommand{\act}{\activationfunction}

\newcommand{\datain}{x}
\newcommand{\x}{\datain}
\newcommand{\xn}[1]{\datain_{#1}}
\newcommand{\xni}[2]{\datain_{#1,#2}}

\newcommand{\datatruth}{y}
\newcommand{\y}{\datatruth}
\newcommand{\yn}[1]{\datatruth_{#1}}
\newcommand{\yni}[2]{\datatruth_{#1,#2}}

\newcommand{\approximations}{\hat{\datatruth}}
\newcommand{\yh}{\approximations}
\newcommand{\yhn}[1]{\approximations_{#1}}
\newcommand{\yhni}[2]{\approximations_{#1,#2}}

\newcommand{\bigo}[1]{\mathcal{O}(#1)}

\newcommand{\nnfunc}{f}
\newcommand{\loss}{\mathcal{L}}

\newcommand{\width}{\mathcal{W}}
\newcommand{\widthk}[1]{\mathcal{W}_{#1}}

\newcommand{\height}{\mathcal{H}}
\newcommand{\heightk}[1]{\mathcal{H}_{#1}}

\newcommand{\patch}{p}
\newcommand{\p}{\patch}
\newcommand{\pk}[1]{\patch^{(#1)}}
\newcommand{\pki}[2]{\patch^{(#1)}_{#2}}

\newcommand{\kernelsize}{K}

\newcommand{\stride}{s}
\newcommand{\s}{\stride}
\newcommand{\sk}[1]{\s_{#1}}

\newcommand{\imagedimsk}[1]{\real{\heightk{#1} \times \widthk{#1} \times \mk{#1} }}

\newcommand{\RELU}{\textrm{ReLU}}
\newcommand{\RMSE}{\textrm{RMSE}}
\newcommand{\CE}{\textrm{CE}}
\newcommand{\SCE}{\textrm{SCE}}

\newcommand*{\thead}[1]{\multicolumn{1}{c}{\bfseries #1}}

\title{
{\textbf{Master's Thesis}\\Faster Convolutional Neural Networks}\\
{\large Radboud University, Nijmegen}
}
\author{Erdi \c{C}all{\i}}

\begin{document}
\maketitle
\chapter*{Abstract}
There exists a gap between the computational cost of state of the art image processing models and the processing power of publicly available devices. This gap is reducing the applicability of these promising models. Trying to bridge this gap, first we investigate some methods to reduce the computational cost of a model. Secondly, we look for alternative operations to design state of the art models. Using these alternative operations, we train a model for the CIFAR-10 classification task. Our model achieves similar results ($91.1\%$ top-1 accuracy) to ResNet-20 ($91.25\%$ top-1 accuracy) with 2 times smaller the model size and 3 times fewer floating point operations. We observe that it is not possible to reduce the computational cost of our model using techniques such as pruning and factorization.
%\chapter*{Dedication}
%To mum and dad

%\chapter*{Declaration}
%I declare that..

%\chapter*{Acknowledgements}
%I want to thank...

\tableofcontents

\chapter{Introduction}
% !TEX root = ../thesis.tex
\iffalse
Introduction
You can't write a good introduction until you know what the body of the paper says. Consider writing the introductory section(s) after you have completed the rest of the paper, rather than before.

Be sure to include a hook at the beginning of the introduction. This is a statement of something sufficiently interesting to motivate your reader to read the rest of the paper, it is an important/interesting scientific problem that your paper either solves or addresses. You should draw the reader in and make them want to read the rest of the paper.

The next paragraphs in the introduction should cite previous research in this area. It should cite those who had the idea or ideas first, and should also cite those who have done the most recent and relevant work. You should then go on to explain why more work was necessary (your work, of course.)
 
What else belongs in the introductory section(s) of your paper? 

    A statement of the goal of the paper: why the study was undertaken, or why the paper was written. Do not repeat the abstract. 
    Sufficient background information to allow the reader to understand the context and significance of the question you are trying to address. 
    Proper acknowledgement of the previous work on which you are building. Sufficient references such that a reader could, by going to the library, achieve a sophisticated understanding of the context and significance of the question.
    The introduction should be focused on the thesis question(s).  All cited work should be directly relevent to the goals of the thesis.  This is not a place to summarize everything you have ever read on a subject.
    Explain the scope of your work, what will and will not be included. 
    A verbal "road map" or verbal "table of contents" guiding the reader to what lies ahead. 
    Is it obvious where introductory material ("old stuff") ends and your contribution ("new stuff") begins? 

Remember that this is not a review paper. We are looking for original work and interpretation/analysis by you. Break up the introduction section into logical segments by using subheads. 
\fi
Recent state of the art Deep Learning models are surpassing previous methods. Fields such as; computer vision, automatic speech recognition, natural language processing, speech recognition, and bioinformatics make use of these models. They use deep models consisting of many layers (e.g. 152 layers in  \cite{He:2015aa}), many parameters in each layer, and as a result, a lot of Floating Point Operations to run Inference (e.g. $11.3 \times 10^9$ in  \cite{He:2015aa}).
In contrast, mobile devices have limited processing power and memory. Also the best practice is to provide a fluent user experience with low response time. Thus, we should change these models to provide a good user experience.
There is research on methods to define optimized models or optimize a given model. These methods consist; pruning unimportant parameters, using less bits to represent parameters, or using less parameters by using more optimized structures. 

In this research we are going run experiments to answer;
\begin{enumerate}
\item    Which models are running slow in Mobile Devices?
\item    Why these models are running slow?
\item    Which methods can we use to optimize these models?
\item    What is the trade off of using these methods?
\item    Why an optimization technique is working or not on a model?
\item    Can we define a more optimized model for the same task?
\item    How can we combine different optimization techniques?
\item    Are these optimized models efficient enough to run in Mobile Devices? 
\end{enumerate}

\section{Recent Studies}
Artificial Neural Networks (ANN) have several parameters such as number of hidden layers, number of neurons in a layer, or the structure of a layer. Until now, we have seen different combinations for these parameters. For example, \cite{Simonyan:2014aa} introduces a model called VGGNet. VGGNet introduces more layers (16 to 19) than the previous models. They show adding more layers effects the accuracy. \cite{He:2015aa} introduces the residual connections. This new connection between layers is capable of stacking more layers than before. Training up to 152 layers, they show superior accuracy. \cite{Zagoruyko:2016aa} compares having higher number of neurons in each layer to having more layers. Each combination resulting in a unique model with a different accuracy level. In contrast to all these, \cite{Szegedy:2014aa} suggests something different. Having a good harmony within the network works better than having more parameters. Supporting that, \cite{Canziani:2016aa} does a detailed comparison of different models. They show that, increasing the number of hidden layers or the number of neurons in a layer does not necessarily increase the accuracy. 

Following these, our hypothesis is, some models are over-parameterized. Meaning they contain parameters that they are not making use of. Therefore, they are making unnecessary computations with them.
\todoin{things to be explained: FC layer, Convolution Operation, what do we mean when we say neuron/node, activation, bias, training dataset, different paddings, SVD}
\todoin{Luc says: Like add a mathematical formulation of deep networks, how you go from NN to DNN to CNN and RNN. What are the pros and cons of these networks? Also, here I would explain all concepts that arise later (for example: what is a conv kernel, SAME padding, etc)}

\section{Dependencies}
\subsection{Tensorflow}
In our research, we will strictly use Tensorflow \cite{abadi2016tensorflow}. \todoin{What is tensorflow, why we chose it, what are the advantages of using it, what are the limitations that come with it}
\subsection{CIFAR10}
\subsection{MNIST}

\chapter{Methods}
% !TEX root = ../thesis.tex
\iffalse
Taken from http://www.ldeo.columbia.edu/~martins/sen_sem/thesis_org.html
Methods
What belongs in the "methods" section of a scientific paper?
    Information to allow the reader to assess the believability of your results.
    Information needed by another researcher to replicate your experiment.
    Description of your materials, procedure, theory.
    Calculations, technique, procedure, equipment, and calibration plots. 
    Limitations, assumptions, and range of validity.
    Desciption of your analystical methods, including reference to any specialized statistical software. 
The methods section should answering the following questions and caveats: 
    Could one accurately replicate the study (for example, all of the optional and adjustable parameters on any sensors or instruments that were used to acquire the data)?
    Could another researcher accurately find and reoccupy the sampling stations or track lines?
    Is there enough information provided about any instruments used so that a functionally equivalent instrument could be used to repeat the experiment?
    If the data are in the public domain, could another researcher lay his or her hands on the identical data set?
    Could one replicate any laboratory analyses that were used? 
    Could one replicate any statistical analyses?
    Could another researcher approximately replicate the key algorithms of any computer software?
Citations in this section should be limited to data sources and references of where to find more complete descriptions of procedures.
Do not include descriptions of results. 
\fi
\section{Dependencies}
\subsection{Tensorflow}
In our research, we will strictly use Tensorflow \cite{abadi2016tensorflow}. \todo[inline]{What is tensorflow, why we chose it, what are the advantages of using it, what are the limitations that come with it}
\todo[inline]{things to be explained: FC layer, Convolution Operation, what do we mean when we say neuron/node, activation, bias, training dataset}
\section{Pruning}
Pruning aims to reduce the number of operations by deleting the parameters that has low or no impact in the result. Studies show that applying this method in an ANN is effective in reducing the model complexity, improving generalization, and they are effective in reducing the required training cycles. In our experiments we will try to reproduce these effects.
To visualize these methods, let's think of two fully connected layers, $\mathbf{l_1}$ and $\mathbf{l_2}$. $\mathbf{l_1}$ is the input of this operation and it consists of $N$ values, $\mathbf{l_1}=(l_{11}, l_{12}, ..., l_{1N})$. $\mathbf{l_2}$ is the output of this operation consists of $M$ values, $\mathbf{l_2}=(l_{21}, l_{22}, ..., l_{2M})$. Between these two layers, there is a weight matrix $W$ with size $N \times M$. The operation, that we want to optimize is, $\mathbf{l_2} = \mathbf{l_1}W$.
\todo[inline]{maybe explain in more detail and give examples of pruning algorithms here. (e.g. Optimal Brain Damage, Second order derivatives for network pruning: Optimal Brain Surgeon, Optimal Brain Surgeon and general network pruning, SEE Pruning Algorithms-a survey from R. Reed)}

\subsection{Pruning Individual Weights}
With this subcategory of pruning algorithms we want to optimize the number of floating point operations by removing some values from $W$. Theoretically, it makes sense to remove individual scalars from W, and exclude operations related to them. This would ideally reduce the required number of floating point operations. But in our library of our choice, Tensorflow, matrix multiplication implementation \texttt{tf.matmul} do not consider such a change. It takes two fixed size matrices, and does the computations using all of their values. Tensorflow also has another matrix multiplication operation, \texttt{tf.sparse\_tensor\_dense\_matmul}. This operation takes a sparse matrix and a dense matrix as inputs and outputs a dense matrix. To implement this method, we could convert W to a sparse tensor after pruning the weights. But, Tensorflow documentations about this method state;
\begin{itemize}
\item Will the SparseTensor A fit in memory if densified?
\item Is the column count of the product large ($>> 1$)?
\item Is the density of A larger than approximately $15\%$?
\end{itemize}
"If the answer to several of these questions is yes, consider converting the SparseTensor to a dense one.". In our terms, SparseTensor A is corresponding to the pruned version of $W$. 

Since $W$ was already dense before, we can assume that the answer to the first question is yes. The column count of our product is $M$ which is much larger than $1$ in some cases. Also we don't know anything about the density of pruned version of $W$. Looking at these facts, we are assuming that implementing this operation will be problematic. Instead of delving deeper into these problems to evaluate this method, we will move on to other methods.

\subsection{Activation Based Pruning - Fully Connected Layers} \label{sec:activation-based-pruning-convolution}
Activation based pruning, works by looking at individual values in layers, and prunes the layer and corresponding weight row/columns completely. To visualize this, we will assume that the fully connected layers we have defined are, trained to some extent, and activated using ReLU activations. With this definition, if we apply our dataset and count the number of activations in $\mathbf{l_1}$ and $\mathbf{l_2}$, we may realize that there are some neurons that are not being activated at all. By removing these neurons from the layers, we can reduce the number of operations. This removal operation is done by removing neurons based on their activations. 

\subsubsection{Experiment Set-up}
To verify the validity of this method, we set up a basic experiment. We have implemented a neural network consisting of 2 inputs, $\mathbf{i} = (i_1, i_2)$, 1 fully connected layer with $n = 1000$ hidden nodes and 1 output, $o$. We have used ReLU \cite{nair2010rectified} activations on our hidden layer. For the sake of simplicity, we have defined the expected output $y$ as $y = i_1 + i_2$. We chose a simple problem so that we precisely know the most optimum neural network structure that would be able to perform this calculation. Which is the same network where the fully connected layer has one hidden node, all weights equal to $1$ and all biases equal to $0$.

We have calculated the loss using mean squared error, and optimized it using Momentum Optimizer (learning rate $0.01$ and momentum $0.3$). Using $1.000.000$ samples, we trained the network with batch size $1000$. With these parameters, we ran a training session with 10 epochs and we have observed that the loss didn't converge to 0. Therefore, the model was unable to find the correct solution with this optimizer. 
\todo[inline]{We should also try different optimizers here}
\subsubsection{Vanilla Pruning}
First we have implemented the very basic idea of pruning unused activations. To do so, we defined training cycles based on the method defined in \cite{Hu:2016aa}. In each training cycle, 1) we have trained the model for some epochs, 2) we lock the weights, 3) feed the training data to the network and count the activations for each neuron in the hidden layer, 4) prune the neurons that have less than or equal to the activation threshold, 5) go back to step $1$ if some neurons were pruned, stop otherwise.

When tested with $0$ activation threshold, after the first training cycle, this method did not to prune more weights. In our experiments, we have pruned approximately 950 weights out of 1000. This result is promising but at the same time, it's not close enough to the result we were expecting. We delved deeper into the source of this issue.

\todo[inline]{We should try different optimizers and make the beginning of the case about why we decided to distort weights.Tell that we have checked the gradients and seen that they were mostly in one direction (+). }

\subsubsection{Distorted Pruning}
When we inspected the gradients of weights, we have seen that most of them were in the positive direction. In our case, this trend in gradients is not helping with the understanding of which neurons are necessary, and which are not. This trend can also be understood as, the feature representation is shared among different hidden neurons. 
\todo[inline]{talk about what does "all gradients are in the positive direction" mean for feature representation}
To prevent shared feature representation, we have decided to push the weights to different directions, randomly. This resulted with the result closes to optimum, but not the exact solution that we were looking for. 
\todo[inline]{the results were in a form not resembling the real solution. maybe because floating point numbers not adding up perfectly, but the result is almost the same in terms of our loss. the exact values of weights and biases are: \\
\texttt{w1: [[ 0.74285096], [ 0.64994317]]\\
b1: [ 7.80925274]\\
w2: [[-6.75151157]]\\
b2: [ 7.80925274]}\\
So since our random values are between -1 and 1, these values are actually okay.}
\todo[inline]{talk about how you decide on the amount of distortion (currently $rand(weights.shape) * (1-var(weights))$). Talk about what changed when we introduced }
\subsubsection{Regularized Distorted Pruning}
Since the solution we found is only resembling our result under some boundaries, we have decided to add an l1 regularizer to our loss. By doing so we are aiming to push the high bias and w2 values closer to 0. But it doesn't really make any difference when used with Moment Optimizer.


\subsection{Activation Based Pruning - Convolution and Deconvolutions}
\todo[inline]{Put references for conv and deconv operations. }
In theory, convolution operation is a matrix multiplication applied on a sliding window. Thus, counting the output feature activations of a convolution operation, we can apply activation based pruning. 

\subsubsection{Experiment Set-up}
To verify the validity of this method, we have implemented an auto encoder for MNIST Dataset \cite{lecun1998mnist}. MNIST contains $28 \times 28$ grayscale images of handwritten digits. The autoencoder consists of two parts. First part is the encoder. The encoder aims to reduce the dimensionality of input. The decoder aims to convert the encoded data back to it's original form.

We have defined the auto encoder with two encoder blocks followed by two decoder blocks. Each encoding block is running convolutions with kernel size $3$, strides of $2$ and $SAME$ padding. Then we are adding bias to this result, following this we are applying batch normalization \cite{ioffe2015batch} and then ReLU activation \cite{nair2010rectified}. Each decoding block is running deconvolutions with kernel size $3$ and strides of $2$. Followed by adding bias, batch normalization and ReLU activations. 

The information contained in one $28 \times 28 \times 1$ matrix is represented with $784$ units (floating points in this case). Therefore, a good autoencoder should be capable of reducing this number when encoding. Similarly, converting the reduced matrix back to it's original form with minimal loss while decoding. Also, the simplest autoencoder should be capable of keeping this number same among blocks while keeping the loss at 0. 

In our case, our encoder blocks are reducing the matrix width and height to half. Therefore, if they output $4$ times the number of input channels, they should represent the same information losslessly. Similarly our decoder blocks are doubling the matrix width and height. Therefore if they output a quarter of the number of input channels, they should be able to decode the encoded information perfectly. In Table \ref{tab:mnist_baseline_encoder} we have defined the layer output dimensions for that baseline auto-encoder.
\begin{table}
\begin{center}
\begin{tabular}{ c | c }
 Block Name & Output Dimensions ($h \times w \times c$) \\
 \hline
 Input Image & $28 \times 28 \times 1$ \\
 Encoder 1 & $14 \times 14 \times 4$ \\  
 Encoder 2 & $7 \times 7 \times 16$ \\
 Decoder 1 & $14 \times 14 \times 4$ \\  
 Decoder 2 & $28 \times 28 \times 1$ 
\end{tabular}
\end{center}
\caption{The baseline network that could perform lossless encoding in theory.}
\label{tab:mnist_baseline_encoder}
\end{table}

To define the network to experiment on, we chose $[32, 64, 32]$ as the output channels of Encoder 1, Encoder 2 and Decoder 1 respectively.

\todo[inline]{This definition may not be good. check it.}

\subsubsection{Vanilla Pruning}
As we did in the Fully Connected Layers, we have pruned the connections that are not being activated. In these experiments we have seen that the network has been pruned insignificantly. After applying this method, we have achieved a network consisting of $[16, 64, 22]$ output channels for blocks Encoder 1, Encoder 2 and Decoder 1 respectively.

\subsubsection{Distorted Pruning}
\todo[inline]{this doesn't change anything.}



\todo[inline]{we can also check activation probabilities and make a decisions based on this data}




\section{Efficient Structures}
\subsection{1D Convolutions}
Convolution kernels could be decomposed into smaller operations. For example using SVD, we can decompose the convolution kernel into the multiplication for two smaller kernels, and by applying these kernels consequently, we can approximate the convolution operation with fewer floating point operations. Instead of doing separating learned convolutions, we are going to use the method introduced in \cite{alvarez2016decomposeme}. This method forces the separability of convolution operation as a hard constraint. 

Normally convolution operations are defined 2 dimensional ($2D$). That is, kernel sizes are ($2D$). For example convolutions we have used in Section \ref{sec:activation-based-pruning-convolution} are $2D$, their kernel sizes are $3 \times 3$. With this method, we are aiming to construct a $N \times N$ convolution operation as a combination two convolution operations. The first convolution has kernel size $1 \times N$ and the following convolution has $N \times 1$, or vice versa. Looking back at the experiment we did in  Section \ref{sec:activation-based-pruning-convolution}, instead of applying one $3 \times 3$ convolution, we are talking about applying one $1 \times 3$ convolution followed by a $3 \times 1$ convolution. 

The amount of speed up can be calculated with Big-O notation($\mathcal{O}$). A convolution operation can be seen as a matrix multiplication applied to consequent subsections of an image. To visualize this, let's assume that we have a convolution operation with kernel size $N$, input channels $K$ and output channels $L$. Assuming that we are applying that operation to a $N \times N$ image patch with $K$ channels, our operation can be simplified as multiplying this $N*N*K$ vector with $N*N*K \times L$ matrix. The complexity of this operation can be expressed as $\mathcal{O}(N^2KL)$. When composed as two $1D$ compositions, this operation will be represented with two $1D$ convolutions. First convolution is multiplying vector $N*K$ with matrix $N*K \times P$ and the second convolution is multiplying vector $N*P$ with matrix $N*P \times L$. As you can see we have introduced the variable $P$ as the intermediate output of the first convolution. The complexity of these operations can be described as, $\mathcal{O}(NLP)$ and $\mathcal{O}(NPK)$. And summing them up, we can say $\mathcal{O}(NP(L + K))$. To determine if this decomposition is making things basic, we could basically try to define a meaningful $P$ while trying to make sure that $\frac{NP(L+K)}{N^2KL} < 1$.
\todo[inline]{shall we cite Big $\mathcal{O}$?}

To show the validity of this method, we have conducted two experiments. 
\subsubsection{Experiment - MNIST Classification}
Our first experiment is to apply this decomposition on a convolutional classifier for MNIST Dataset. This classifier is originally consisting of three Convolutional Layers and one Fully Connected layer. This configuration is defined in Table \ref{tab:nxn-mnist-classifier}. By converting the 
\begin{table}
\centering
\begin{tabular}{l | c | c}
Layer & Configuration & Output\\
\hline
Input Image & & $28 \times 28 \times 1$ \\
\hline
Convolution & \small $N=5, \text{strides}=1, \text{padding}=SAME$ & $28 \times 28 \times 32$ \\
Add Bias & & $28 \times 28 \times 32$ \\
ReLU & & $28 \times 28 \times 32$ \\
Max Pool & size=$ 2 \times 2$ & $14 \times 14 \times 32$ \\
\hline
Convolution & \small $N=5, \text{strides}=1, \text{padding}=SAME$ & $14 \times 14 \times 64$ \\
Add Bias & & $14 \times 14 \times 64$ \\
ReLU & & $14 \times 14 \times 64$ \\
Max Pool & size=$ 2 \times 2$ & $7 \times 7 \times 64$ \\
\hline
Convolution & \small $N=7, \text{strides}=1, \text{padding}=VALID$ & $1 \times 128$ \\
Add Bias & & $ 1 \times 128$ \\
ReLU & & $ 1 \times 128$ \\
\hline
FC Layer &  & $1 \times 10$ \\
Add Bias & & $1 \times 10$ 
\end{tabular}
\caption{Network configuration for MNIST Classifier, output of every row is applied as the input of next.}
\label{tab:nxn-mnist-classifier}
\end{table}

\subsection{Bottleneck Architecture}
\cite{He:2015aa} introduced residual connections with bottleneck blocks. To optimize the performance of their network, they have introduced the bottleneck blocks. These blocks start with a $ 1 \times 1$ convolution that scales down the number of features to half. Followed by a $3 \times 3$ convolution without changing the number of features, and following that with a $1 \times 1$ convolution to quadruple the number of features. They also introduce residual connections, but we will not talk about these now.

To compare the effectiveness of bottleneck blocks, \cite{He:2015aa} have compared the 34-layer structure consisting of pairs of $3 \times 3$ blocks with the 54-layer structure consisting of bottleneck blocks. Both networks have very similar number of FLOPS but the 54-layer network clearly outperforms the 30-layer network. This is a very clear demonstration of the effectiveness of bottleneck blocks.

\begin{table}[]
\centering
\begin{tabular}{ | c | c | c | c | }
\hline
layer name			& output size 					& 34-layer																& 50-layer																			\\ \hline
conv1				& $112 \times112$				& \multicolumn{2}{c|}{$ 7 \times 7$, $64$, stride $2$}																												\\ \hline
\multirow{2}{*}{conv2\_x}	& \multirow{2}{*}{$56 \times 56$} 	& \multicolumn{2}{c|}{$3 \times 3$ max pool, stride $2$}																											\\ \cline{3-4} 
					&							& $\begin{bmatrix} 3 \times 3, &   64 \\ 3 \times 3, &   64 \end{bmatrix} \times 3 $		& $\begin{bmatrix}1 \times 1, & 64 \\ 3 \times 3, & 64 \\ 1 \times 1, & 256 \end{bmatrix}^{} \times 3 $ 		\\ \hline
conv3\_x				& $28 \times 28$				& $\begin{bmatrix} 3 \times 3, & 128 \\ 3 \times 3, & 128 \end{bmatrix} \times 3 $		& $\begin{bmatrix}1 \times 1, & 128 \\ 3 \times 3, & 128 \\ 1 \times 1, & 512 \end{bmatrix} \times 3$		\\ \hline
conv4\_x				& $14 \times 14$				& $\begin{bmatrix} 3 \times 3, & 256 \\ 3 \times 3, & 256 \end{bmatrix} \times 3 $		& $\begin{bmatrix}1 \times 1, & 256 \\ 3 \times 3, & 256 \\ 1 \times 1, & 1024 \end{bmatrix} \times 3$		\\ \hline
conv5\_x				& $  7 \times   7$				& $\begin{bmatrix} 3 \times 3, & 512 \\ 3 \times 3, & 512 \end{bmatrix} \times 3 $		& $\begin{bmatrix}1 \times 1, & 512 \\ 3 \times 3, & 512 \\ 1 \times 1, & 2048 \end{bmatrix} \times 3$		\\ \hline
					& $  1 \times   1$				&\multicolumn{2}{c|}{average pool, 1000-d fc, softmax}																											\\ \hline
\multicolumn{2}{| c |}{FLOPs}							& $3.6 \times 10^9$														& $3.8 \times 10^9$																	\\ \hline
\multicolumn{2}{| c |}{top-1 err}							& $21.53$																& $20.74$																			\\ \hline
\multicolumn{2}{| c |}{top-5 err}							& $5.60$																& $5.25$																			\\ \hline
\end{tabular}
\caption{Comparison of bottleneck blocks (50-layer) with stacked $ 3 \times 3$ layers (34-layer). }
\label{tab:bottleneck-comparison}
\end{table}






























\chapter{Results}
% !TEX root = ../thesis.tex
\iffalse
Results

    The results are actual statements of observations, including statistics, tables and graphs.
    Indicate information on range of variation.
    Mention negative results as well as positive. Do not interpret results - save that for the discussion. 
    Lay out the case as for a jury. Present sufficient details so that others can draw their own inferences and construct their own explanations. 
    Use S.I. units (m, s, kg, W, etc.) throughout the thesis. 
    Break up your results into logical segments by using subheadings
    Key results should be stated in clear sentences at the beginning of paragraphs.  It is far better to say "X had significant positive relationship with Y (linear regression p<0.01, r^2=0.79)" then to start with a less informative like "There is a significant relationship between X and Y".  Describe the nature of the findings; do not just tell the reader whether or not they are significant.  
\fi



\chapter{Discussion}
% !TEX root = ../thesis.tex

\iffalse
 Start with a few sentences that summarize the most important results. The discussion section should be a brief essay in itself, answering the following questions and caveats: 

    What are the major patterns in the observations? (Refer to spatial and temporal variations.)
    What are the relationships, trends and generalizations among the results?
    What are the exceptions to these patterns or generalizations?
    What are the likely causes (mechanisms) underlying these patterns resulting predictions?
    Is there agreement or disagreement with previous work?
    Interpret results in terms of background laid out in the introduction - what is the relationship of the present results to the original question?
    What is the implication of the present results for other unanswered questions in earth sciences, ecology, environmental policy, etc....?
    Multiple hypotheses: There are usually several possible explanations for results. Be careful to consider all of these rather than simply pushing your favorite one. If you can eliminate all but one, that is great, but often that is not possible with the data in hand. In that case you should give even treatment to the remaining possibilities, and try to indicate ways in which future work may lead to their discrimination.
    Avoid bandwagons: A special case of the above. Avoid jumping a currently fashionable point of view unless your results really do strongly support them. 
    What are the things we now know or understand that we did not know or understand before the present work?
    Include the evidence or line of reasoning supporting each interpretation.
    What is the significance of the present results: why should we care? 

This section should be rich in references to similar work and background needed to interpret results. However, interpretation/discussion section(s) are often too long and verbose. Is there material that does not contribute to one of the elements listed above? If so, this may be material that you will want to consider deleting or moving. Break up the section into logical segments by using subheads. 
\fi

Here we will try to criticize our decisions in model and method selection which are increasing the questionability of our results. We will also talk about the problems we have faced during this research. 

We think that some of our design decisions regarding the pruning experiments were wrong or misleading. First, to select the best nodes to prune, we tried to make use of some pruning criteria. However, in our experiments, we have seen that randomly pruning nodes and leaving out some nodes also produced similar results. We think that this is the result of using high learning rates in the beginning of every training cycle. To fix that, we could use smaller learning rates in every training cycle, but this would make our model converge to a non-optimal structure. Second, we started with very large initial models. We believe that this may have left a false impression that pruning methods will reduce the model complexity by 1000 times. We think that this also is the case for some recent work on pruning. Given a sufficiently large model and a sufficiently small problem, creating a desired pruning rate is possible. We think that pruned models should also be compared with other small models defined for the task. Third, our pruning criteria selection was based on very simple methods. However, more complicated methods, such as aforementioned \textit{frequency sensitive hashing} or some gradient based pruning criteria may have worked better on small models. 

Also when we were applying the pruning techniques on small models, we may have made some bad decisions. First, while pruning residual networks, we have set a very strict rule, that grouped separable residual blocks and pruned their output features together. This is a very strong assumption. We could also define a rule to prune residual blocks separately and place zeros on pruned indices before residual connections with previous layers. However, we decided to stay away from this complicated method. Second, the autoencoder model that we have experimented with is not a classification model. However, we have applied the best practices that worked for this model to prune a classification model. Instead of experimenting with an autoencoder, we should have worked with a classifier. Third, since residual blocks have a one to one relationship between their output channels it may have been possible to prune residual blocks in some cases. However, we didn't have time to try such a method.

In Section \ref{sec:conv_alternatives}, we have defined a neural network to compare alternative convolution operations. Before our experiments, we have tried to find the best settings for some parameters, such as learning rate, regularization constant and optimizer. We think that using the same settings may have influenced our results. Especially because the number of parameters reduce considerably when we use alternative operations. This may have worked in favor of alternative methods if we made a bad choices in selecting these parameters. However, one can also argue that same parameters should be used in such a comparison. Also, we have used large ($5 \times 5$) kernels for these experiments. We are not sure if these results can be translated to $3 \times 3$ kernels. However, to be able to compare kernel composing convolutions with others, we felt an urge to use larger kernels. We could have separated separable convolution and kernel composing convolution experiments and compared them separately with convolution operations using relatively large kernels. 

In Section~\ref{sec:result_models}, we have compared our CIFAR-10 model with ResNet-20 (\cite{He:2015aa}). However there are small differences between how these models were trained. First ResNet-20 is trained using a different data preprocessing and augmentation technique. Second, ResNet-20 (\cite{He:2015aa}) does not employ full pre-activation residual connections. ResNet-20 may have performed better with these techniques.

Compared to MNIST and CIFAR-10, ImageNet is a very large dataset. Using CIFAR-10, we were able to search the parameter space for the smallest model that perform well. However, since training a model for ImageNet takes about a week with the available equipment, we were unable find the best model. Also, we think that our way of aggressively reducing image dimensions was very experimental, and probably unstable. More experiments are required to prove this configuration valid or invalid.

\subsubsection{Using Tensorflow}
We have been using latest versions of Tensorflow. It comes with some advantages, such as:
\begin{itemize}
\item We do not need to implement lower level operations (such as convolutions). It gives us the opportunity to focus on higher level implementations, such as pruning, or factorization. 
\item Most of the operations are highly optimized for many platforms and devices. If we were to implement a model in C++, we'd have to implement it twice, one for training in GPU and another for running in the mobile device. 
\item Tensorflow provides the necessary tools to deploy models on mobile devices.
\end{itemize}

And it comes with some disadvantages, such as:

\begin{itemize}
\item When we started our work, Tensorflow was in version 0.10. By the date we write this, it is on 1.2. There have been 4 major releases that we had to modify our codebase for.
\item Not all operations are properly implemented. For example, before version 1.2, Tensorflow implementation of separable convolutions were not very well optimized. They were as fast as convolution operations. Before that we could only hope that they would optimize their implementation.
\item It is difficult to implement operations (e.g. ef operator \cite{afrasiyabi2017energy}) or play around with existing ones. The documentation describing C++ internals and build procedures (as of Tensorflow 1.2) are not good enough. 
\item Tensorflow does not provide tools to implement low-bit variables (e.g. a 2-bit integer). So it is not possible to implement some methods that make use of variable width decimals. This limitation makes some methods impossible to use or useless. For example it is not possible to use methods that represent weights using variable width decimals. Also, storing low bit weight indices in combination with a small global weight array to reduce the model size is useless. Since we can not use low bit integers to represent these indices, our model size does not shrink at all.
\end{itemize}


\chapter{Conclusion}
% !TEX root = ../thesis.tex

\iffalse
    What is the strongest and most important statement that you can make from your observations? 
    If you met the reader at a meeting six months from now, what do you want them to remember about your paper? 
    Refer back to problem posed, and describe the conclusions that you reached from carrying out this investigation, summarize new observations, new interpretations, and new insights that have resulted from the present work.
    Include the broader implications of your results. 
    Do not repeat word for word the abstract, introduction or discussion.
\fi

In this research, we have investigated some methods to reduce the computational cost of convolutional neural networks. To do that, we experimented with some methods that could be used to define models with lower computational cost. We also experimented with some methods to reduce the computational complexity of a given model. 

To be able to experiment with pruning using larger models, we have implemented a tool to describe pruning routines. We have also implemented a tool that applies simple quantization, pruning and factorization methods to trained models. Using these tools, we have observed that these methods reduce the computational cost of sufficiently large models.

In our experiments we have observed that the models using separable convolutions with non-linearity results with a slightly better accuracy compared to models using convolution or kernel compositing convolution operations. Using them, we have redefined residual blocks and designed a model that achieves similar results to ResNet-20 on CIFAR-10 classification task. Our model is two times wider, however it has less residual blocks, using two times less parameters and requiring 3 times less operations. However, more work needs to be done to achieve similar results using ImageNet dataset.

When developing models aimed for processing power restricted environments, we think that designing and training small models based on the requirements is a more stable alternative to compressing large networks. We have seen that wider and shallower residual networks using separable residual blocks are one way of designing such models.


\bibliographystyle{alpha}
\bibliography{references}

\end{document}
