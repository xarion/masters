\documentclass[12pt]{report}

\usepackage[english]{babel}
\usepackage[utf8x]{inputenc}
\usepackage{amsmath}
\usepackage{graphicx}
\usepackage[colorinlistoftodos]{todonotes}
\usepackage{hyperref}
\usepackage{multirow}
\usepackage{array,booktabs}
\usepackage{tikz}
\usetikzlibrary{arrows}
\usepackage{csquotes}
\usepackage{amssymb}
\usepackage{epstopdf}
\usepackage{wrapfig}
\usepackage{mathtools}

\DeclarePairedDelimiter\ceil{\lceil}{\rceil}
\DeclarePairedDelimiter\floor{\lfloor}{\rfloor}
\usepackage{pdfpages}
\usepackage{subcaption}

\newcommand{\todoin}{\todo[inline]}
\newcommand{\realR}{\mathbb{R}}
\newcommand{\real}[1]{\realR^{#1}}
\newcommand{\inreal}[1]{\in \real{#1}}

\newcommand{\weights}{w}
\newcommand{\w}{\weights}
\newcommand{\wk}[1]{\weights^{(#1)}}
\newcommand{\wki}[2]{\wk{#1}_{#2}}

\newcommand{\all}[1]{\expandafter\MakeUppercase\expandafter{#1}}
\newcommand{\num}[1]{\expandafter\MakeUppercase\expandafter{#1}}

\newcommand{\layer}{l}
\newcommand{\lk}[1]{\layer^{(#1)}}
\newcommand{\lki}[2]{\layer^{(#1)}_{#2}}

\newcommand{\numnodes}{m}
\newcommand{\m}{\numnodes}
\newcommand{\mk}[1]{\numnodes^{(#1)}}
\newcommand{\mki}[2]{\numnodes^{(#1)}_{#2}}

\newcommand{\outputvar}{o}
\newcommand{\ok}[1]{\outputvar^{(#1)}}
\newcommand{\oki}[2]{\outputvar^{(#1)}_{#2}}
\newcommand{\okT}[1]{\outputvar^{(#1)T}}
\newcommand{\okiT}[2]{\outputvar^{(#1)T}_{#2}}


\newcommand{\biasterm}{b}
\newcommand{\bk}[1]{\biasterm^{(#1)}}
\newcommand{\bki}[2]{\biasterm^{(#1)}_{#2}}

\newcommand{\layerfunction}{\psi}

\newcommand{\lf}{\layerfunction}
\newcommand{\lfp}{\lf'}
\newcommand{\lft}[1]{\lf^{(#1)}}
\newcommand{\lfk}[1]{\lf_{(#1)}}
\newcommand{\lftk}[2]{\lf^{(#1)}_{(#2)}}
\newcommand{\lfkt}[2]{\lf_{(#1)}^{(#2)}}
\newcommand{\lfpk}[1]{\lf'_{(#1)}}
\newcommand{\lfpkt}[2]{\lf_{(#1)}'^{(#2)}}

\newcommand{\FC}{\lft{FC}}
\newcommand{\FCk}[1]{\FC_{(#1)}}

\newcommand{\lftp}[1]{\lf'^{(#1)}}
\newcommand{\FCp}{\lftp{FC}}
\newcommand{\FCkp}[1]{\FCp_{(#1)}}

\newcommand{\conv}{\lft{Conv}}
\newcommand{\convk}[1]{\conv_{(#1)}}

\newcommand{\convp}{\lftp{Conv}}
\newcommand{\convkp}[1]{\convp_{(#1)}}

\newcommand{\maxpool}{\lft{maxpool}}
\newcommand{\maxpoolk}[1]{\maxpool_{(#1)}}

\newcommand{\avgpool}{\lft{avgpool}}
\newcommand{\avgpoolk}[1]{\avgpool_{(#1)}}

\newcommand{\activationfunction}{\sigma}
\newcommand{\act}{\activationfunction}

\newcommand{\datain}{x}
\newcommand{\x}{\datain}
\newcommand{\xn}[1]{\datain_{#1}}
\newcommand{\xni}[2]{\datain_{#1,#2}}

\newcommand{\datatruth}{y}
\newcommand{\y}{\datatruth}
\newcommand{\yn}[1]{\datatruth_{#1}}
\newcommand{\yni}[2]{\datatruth_{#1,#2}}

\newcommand{\approximations}{\hat{\datatruth}}
\newcommand{\yh}{\approximations}
\newcommand{\yhn}[1]{\approximations_{#1}}
\newcommand{\yhni}[2]{\approximations_{#1,#2}}

\newcommand{\bigo}[1]{\mathcal{O}(#1)}

\newcommand{\nnfunc}{f}
\newcommand{\loss}{\mathcal{L}}

\newcommand{\width}{\mathcal{W}}
\newcommand{\widthk}[1]{\mathcal{W}_{#1}}

\newcommand{\height}{\mathcal{H}}
\newcommand{\heightk}[1]{\mathcal{H}_{#1}}

\newcommand{\patch}{p}
\newcommand{\p}{\patch}
\newcommand{\pk}[1]{\patch^{(#1)}}
\newcommand{\pki}[2]{\patch^{(#1)}_{#2}}

\newcommand{\kernelsize}{K}

\newcommand{\stride}{s}
\newcommand{\s}{\stride}
\newcommand{\sk}[1]{\s_{#1}}

\newcommand{\imagedimsk}[1]{\real{\heightk{#1} \times \widthk{#1} \times \mk{#1} }}

\newcommand{\RELU}{\textrm{ReLU}}
\newcommand{\RMSE}{\textrm{RMSE}}
\newcommand{\CE}{\textrm{CE}}
\newcommand{\SCE}{\textrm{SCE}}

\newcommand*{\thead}[1]{\multicolumn{1}{c}{\bfseries #1}}

\title{
{\textbf{Master's Thesis}\\Faster Convolutional Neural Networks}\\
{\large Radboud University, Nijmegen}
}
\author{Erdi \c{C}all{\i}}

\begin{document}
\maketitle
\chapter*{Abstract}
There exists a gap between the computational cost of state of the art image processing models and the processing power of publicly available devices. This gap is reducing the applicability of these promising models. Trying to bridge this gap, first we investigate some methods to reduce the computational cost of a model. Secondly, we look for alternative operations to design state of the art models. Using these alternative operations, we train a model for the CIFAR-10 classification task. Our model achieves similar results ($91.1\%$ top-1 accuracy) to ResNet-20 ($91.25\%$ top-1 accuracy) with 2 times smaller the model size and 3 times fewer floating point operations. We observe that it is not possible to reduce the computational cost of our model using techniques such as pruning and factorization.
%\chapter*{Dedication}
%To mum and dad

%\chapter*{Declaration}
%I declare that..

%\chapter*{Acknowledgements}
%I want to thank...

\tableofcontents

\chapter{Introduction}
% !TEX root = ../thesis.tex

\iffalse
\subsection{Activation Based Pruning - Fully Connected Layers} \label{sec:activation-based-pruning-convolution}
Activation based pruning, works by looking at individual values in layers, and prunes the layer and corresponding weight row/columns completely. To visualize this, we will assume that the fully connected layers we have defined are, trained to some extent, and activated using ReLU activations. With this definition, if we apply our dataset and count the number of activations in $\mathbf{l_1}$ and $\mathbf{l_2}$, we may realize that there are some neurons that are not being activated at all. By removing these neurons from the layers, we can reduce the number of operations. This removal operation is done by removing neurons based on their activations. 
\todoin{add figure to show what happens when we prune}

\subsection{Activation Based Pruning - Convolution and Deconvolutions}
\todoin{Put references for conv and deconv operations. }
In theory, convolution operation is a matrix multiplication applied on a sliding window. Thus, counting the output feature activations of a convolution operation, we can apply activation based pruning. 

\subsection{Second Order Derivatives (Fischer Information Matrix)}
\fi


\iffalse
Introduction
You can't write a good introduction until you know what the body of the paper says. Consider writing the introductory section(s) after you have completed the rest of the paper, rather than before.

Be sure to include a hook at the beginning of the introduction. This is a statement of something sufficiently interesting to motivate your reader to read the rest of the paper, it is an important/interesting scientific problem that your paper either solves or addresses. You should draw the reader in and make them want to read the rest of the paper.

The next paragraphs in the introduction should cite previous research in this area. It should cite those who had the idea or ideas first, and should also cite those who have done the most recent and relevant work. You should then go on to explain why more work was necessary (your work, of course.)
 
What else belongs in the introductory section(s) of your paper? 

    A statement of the goal of the paper: why the study was undertaken, or why the paper was written. Do not repeat the abstract. 
    Sufficient background information to allow the reader to understand the context and significance of the question you are trying to address. 
    Proper acknowledgement of the previous work on which you are building. Sufficient references such that a reader could, by going to the library, achieve a sophisticated understanding of the context and significance of the question.
    The introduction should be focused on the thesis question(s).  All cited work should be directly relevent to the goals of the thesis.  This is not a place to summarize everything you have ever read on a subject.
    Explain the scope of your work, what will and will not be included. 
    A verbal "road map" or verbal "table of contents" guiding the reader to what lies ahead. 
    Is it obvious where introductory material ("old stuff") ends and your contribution ("new stuff") begins? 

Remember that this is not a review paper. We are looking for original work and interpretation/analysis by you. Break up the introduction section into logical segments by using subheads. 
\fi

The state of the art in Image Processing has changed when graphics processing units (GPU) were used to train neural networks. GPUs contain many cores, they have very large data bandwidth and they are optimized for efficient matrix operations. In 2012, \cite{krizhevsky2012imagenet} won the ImageNet Large Scale Visual Recognition Competition (ILSVRC) classification task (\cite{deng2012image}). They used two GPUs to train an 8 layer convolutional neural network (CNN). Their model has improved the previous (top-5) classification accuracy record from $\sim 74\%$ to $\sim 84\%$. This caused a big trend shift in Computer Vision. 

As the years pass, GPUs got more and more powerful. In 2012, \cite{krizhevsky2012imagenet} used GPUs that had 3 GB memory. Today there are GPUs with up to 12 GB memory. The number of floating point operations per second (FLOPs) has also increased from $2.5$ tera FLOPs (TFLOPs) to $12$ TFLOPs. This gradual but steep change has allowed the use of more layers and more parameters. For example, \cite{Simonyan:2014aa} introduced a model called VGGNet. Their model used up to 19 layers and shown that adding more layers affects the accuracy. \cite{He:2015aa} introduced a new method called residual connections, that allowed the use of up to 200 layers. Building up on such models, in 2016 ILSVRC winning (top-5) classification accuracy is increased to $\sim 97\%$. 

In contrast, \cite{Szegedy:2014aa} have shown that having a good harmony within the network worked better than having more parameters. It has been supported by \cite{Canziani:2016aa}. They have shown the relation between number of parameters of a model and its top-1 classification accuracy in ILSVRC dataset. According to their report, 48 layer Inception-v3 (\cite{Szegedy_2016_CVPR}) provides better top-1 classification accuracy than 152 layer ResNet (\cite{He:2015aa}). They also show that Inception-v3 requires fewer number of floating point operations to compute results. Therefore, revealing that of providing more layers and parameters would not yield better results. 

ILSVRC is one of the most famous competitions in Image Processing. Every year, the winners of this competition are a driving the research on the field. But this competition is not considering the competitive value of limiting number of operations. If we look at the models of 2016 competitors, we see that they use ensembles of models\footnote{\url{http://image-net.org/challenges/LSVRC/2016/results\#team}}. These ensembles are far from being usable in real life because they require a great amount of operations per inference. Not mentioning the number of operations from the result is misleading for the AI community and the public. It creates an unreal expectation that these models are applicable in real life. In this thesis, we want to come up with a state of the art solution that requires a low number of floating point operations per inference. Therefore, bridging the gap between expectations and reality. We will answer,
\begin{quote}
How can we reduce the inference complexity of Neural Networks?
How do these modifications effect accuracy?
\end{quote}
First, we will briefly describe neural networks and some underlying concepts. We will mention the complexities of necessary operations. Then, we will provide known solutions to reduce these complexities.

\section{Neural Networks}
In this chapter, we will try to describe neural networks briefly. We will provide some terminology and give some examples. 

Neural networks are \textit{weighted graphs}. They consist of an ordered set of \textit{layers}, where every layer is a set of \textit{nodes}. The first layer of the neural network is called the \textit{input layer}, and the last one is called the \textit{output layer}. The layers in between are called \textit{hidden layers}. In our case, nodes belonging to one layer are connected to the nodes in the following and/or the previous layers. These connections are weighted edges, and they are mostly called as \textit{weights}. 

Given an input, neural network nodes have \textit{outputs}, which are real numbers. The output of a node is calculated by applying a function ($\psi$) the outputs of the nodes belonging to previous layers . Preceding that, the output of the input layer is calculated using the input data (see Eq. \ref{eq:output_of_layers}).  By calculating the layer outputs consecutively we calculate the output of the output layer. This process is called \textit{inference}. We use the following notations to denote the concepts that we just explained.
\begin{equation}
\label{eq:variable_definitions}
\begin{split}
l_k & \text{: a column vector of nodes for layer $k$}\\
m_k & \text{: the number of nodes in $l_k$}\\
l_{k,i}  & \text{: node $i$ in $l_k$}\\
o_{k}  & \text{: the output vector representing the outputs of nodes in $l_{k}$}\\
o_{k,i}  & \text{: the output of $l_{k,i}$}\\
\mathbf{w}^{(k)}  & \text{: weight matrix connecting nodes in $l_{k-1}$ to nodes in $l_{k}$} \\
w^{(k)}_{i,j}  & \text{: the weight connecting nodes $l_{(k-1),i}$ and $l_{k,j}$} \\
\mathbf{b}^{(k)}  & \text{: the bias term for $l_{k}$} \\
\psi_k & \text{: function to determine $o_k$ given $o_{k-1}$}\\
\sigma & \text{: activation functions} \\
\mathbf{x} & \text{: all inputs of the dataset, consisting of $N$ data points} \\
\mathbf{y} & \text{: all outputs of the dataset} \\
\mathbf{\hat y} & \text{: approximation of the output}  \\
x_n & \text{: $n$th input data ($0 < n \leq N$)} \\
y_n & \text{: $n$th output data ($0 < n \leq N$)} \\
\hat y_n & \text{: approximation of $y_n$ given $x_n$ ($0 < n \leq N$)}\\
\text{FC} & \text{: stands for Fully Connected (e.g. $\psi^{(FC)}$)}
\end{split}
\end{equation}
Therefore the structure of a neural network is determined by the number of layers and the functions that determine the outputs of layers.
\begin{equation}
\label{eq:output_of_layers}
    o_k = 
\begin{cases}
    \psi(o_{k-1}), &\text{if } k\geq 1\\
    \mathbf{x},& k = 0\\
\end{cases}
\end{equation}

\subsection{Fully Connected Layers}
As the name suggests, for two consecutive layers to be \textit{fully connected}, all nodes in the previous layer must be connected to all nodes in the following layer. 

Let's assume two consecutive layers, $l_{k-1} \in \mathbb{R}^{m_{k-1} \times 1}$ and $l_{k} \in \mathbb{R}^{m_k \times 1}$. For these layers to be fully connected, the weight matrix connecting them would be defined as $\mathbf{w}^{(k)} \in \mathbb{R}^{m_{k-1} \times m_{k}}$. Most fully connected layers also include a bias term for every node in $l_k$ ($\mathbf{b}^{(k)} \in  \mathbb{R}^{m_{k}}$). In fully connected layers, $o_k$ would simply be calculated using layer function $\psi^{(FC)}$.
$$ \psi^{(FC)}_k(o_{k-1}) = o_{k-1}^T\mathbf{w}^{(k)} + \mathbf{b}^{(k)}$$
Therefore the computational complexity of $\psi^{(FC)}$ would become
$$\mathcal{O}(\psi^{(FC)}_k) = \mathcal{O}(m_{k-1}m_{k})$$

\subsection{Activation Function and Nonlinearity}
By stacking fully connected layers, we can increase the depth of a neural network. By doing so we want to increase approximation quality of the neural network. However, the $\psi^{(FC)}$ we have defined is a linear function. Therefore if we stack multiple fully connected layers we would end up with a linear model. 

To achieve non-linearity, we apply \textit{activation functions} to the results of $\psi$. There are many activation functions (such as $tanh$ or $sigmoid$) but one very commonly used activation function is $ReLU$ \cite{nair2010rectified}.  
\begin{equation}
\label{eq:relu_definition}
    ReLU(x) = 
\begin{cases}
    x, & \text{if }x \geq 0\\
    0 &  \text{otherwise }\\
\end{cases}
\end{equation}
Therefore we will redefine the fully connected $\psi^{(FC)}$ as;

$$ \psi^{(FC)}_k(o) = \sigma(o^T\mathbf{w}^{(k)} + \mathbf{b}^{(k)})$$

The activation functions does not strictly belong to the definition of fully connected layers. But for simplicity, we are going to include them in the layer functions ($\psi$).

$\psi^{(FC)}$ is one of the most basic building blocks of any Neural Network. Stacking $\mathcal{K}$ of them after the input, we can try to approximate an output given an input. To do that we will calculate the outputs of every layer, starting from the input. 
$$ \mathbf{o} = \{\psi_{k}(o_{k-1}) | k \in [1, \ldots, \mathcal{K}|] \} $$

\subsection{Loss}

To represent the quality of an approximation, we are going to use a loss (or cost) function. A good example would be the loss of a salesman. Assuming a customer who would pay at most \$10 for a given product, if the salesman sells this product for \$4, the salesman would face a loss of \$6 from his potential profit. Or if the salesman tries to sell this product for \$14, the customer will not purchase it and he will face a loss of \$10. In this example, the salesman would want to minimize the loss to earn as much as possible. There are two common properties of loss functions. First, loss is never negative. Second, if we compare two approximations, the one with a smaller loss is better.

\subsubsection{Root Mean Square Error}
A commonly used loss function is Root Mean Square Error (RMSE). Given an approximation ($\hat y \in \mathbb{R}^N$) and the expected output ($y \in \mathbb{R}^N$), RMSE can be calculated as,
\begin{equation*}
\mathcal{L} = RMSE(\hat y, y) = \sqrt{\frac{\sum^N_{n=1} (\hat y_n - y_n)^2 }{N}}
\end{equation*}

\subsubsection{Softmax Cross Entropy}
Another commonly used loss function is Softmax cross entropy (SCE). Softmax cross entropy is used with classification tasks where we are trying to find the class that our input belongs to. Softmax cross entropy first calculates the class probabilities given the input.

Given an approximation ($\hat y \in \mathbb{R}^N$) it first calculates the class probabilities as, 
$$p(i | \hat y_n) = \frac{e^{\hat y_i}}{\sum_{n=1}^{N}e^{\hat y_n}}$$
Then comparing it with the the expected output ($y \in \mathbb{R}^N$), SCE loss can be calculated as,
$$\mathcal{L} = CE(\hat y, y) = - \sum_{n=1}^{N} y_{n} \cdot log(p(i | \hat y_n))$$

\todoin{this definition could be better}

\subsection{Stochastic Gradient Descent}
To provide better approximations, we will try to optimize the neural network parameters. One common way to optimize these parameters is to use Stochastic Gradient Descent (SGD). SGD is an iterative learning method that starts with some initial (random) parameters. Given $\theta \in (\mathbf{w} \cup \mathbf{b}$) to be a parameter that we want to optimize. The learning rule updating theta would be;

$$ \theta = \theta- \eta \nabla_\theta{\mathcal{L}(f(x), y)} $$

where $\eta$ is the learning rate, and $\nabla_\theta{\mathcal{L}(f(x), y)}$ is the partial derivative of the loss in terms of given parameter, $\theta$. One iteration is completed when we update every parameter for given example(s). By performing many iterations, SGD aims to find a global minimum for the loss function, given data and initial parameters.

There are several other optimizers that work in different ways. We will be using Adam Optimizer (\cite{kingma2014adam}), Momentum Optimizer (\cite{qian1999momentum}) and SGD.

\iffalse
\subsection{Training a Neural Network}
Training is the process of learning the best weights given a set of samples. When describing the training process of a model, we will define the configuration we have used. This configuration consists of number of Epochs, batch size, training/validation datasets, loss, optimizer and finally network configuration. 
We will start by defining the network configuration. The network configuration is basically setting up the neural network graph and its operations. Starting from the input nodes we define every layer up to the output layer. To do that, first we need to define our training dataset because the training dataset defines the shape of our input and output layers. Then we will define a loss for the output layer. Using this loss we will define our optimizer. 
\fi

\subsection{Convolutional Layer}
So far we have seen the key elements we can use to create and train fully connected neural networks. To be able to apply neural networks to image inputs (or an at least 2 dimensional data), we can use convolutional layers and convolution operation. Please note that we are assuming one or two dimensional convolutions with same padding. 

Let's assume a 3 dimensional layer output $o_{k-1} \in \mathbb{R}^{H_{k-1} \times W_{k-1} \times m_{k-1}}$ where the dimensions $W_{k-1}$ representing the width, $H_{k-1}$ representing height and $m_{k-1}$ representing number of nodes. Convolution operation first creates a sliding window that goes through width and height. The contents of this sliding window would be patches ($p_{k-1,(I,J)} \in \mathbb{R}^{K \times K \times m_{k-1}}$). By multiplying a weight matrix $\mathbf{w}^{(k)} \in \mathbb{R}^{K \times K \times m_{k-1} \times m_k}$ with every patch, we create a set of output nodes $o_{(k+1),(I,J)} \in \mathbb{R}^{1 \times m_k}$. Those output nodes represent the features at location $(I,J)$. By performing this operation for every patch, we calculate the outputs of a convolutional layer. While calculating the patches, we also make use of a parameter called stride, $s_k \in \mathbb{N}^+$. $s$ defines the number of vertical and horizontal indexes between each patch.

$$ W_k = \floor*{\frac{W_{k-1}}{s_k}}$$
$$ H_k = \floor*{\frac{H_{k-1}}{s_k}}$$
$$ \psi_{k}^{(Conv)} : \mathbb{R}^{H_{k-1} \times W_{k-1} \times m_{k-1} } \rightarrow \mathbb{R}^{ H_k \times W_k \times m_k} $$
$$ p_{k-1,(I,J)} \in \mathbb{R}^{K \times K \times m_{k-1}} $$
$$ p_{k-1,(I,J)} \subset o_{k-1}$$
$$ p_{k-1,(I,J)} = (p_{k-1,(I,J),(i,j)}) $$
$$ p_{k-1,(I,J),(i,j)} \in \mathbb{R}^{m_k-1} $$
$$ p_{k-1,(I,J),(i,j)} = o_{k-1,(a,b)} $$
where
$$ o_{k-1} = (o_{k-1,(a,b)})$$
$$ a = Is + (i - \floor*{K/2}) $$
$$ b = Js + (j - \floor*{K/2}) $$
and 
$$ 0 < I \leq H_k$$
$$ 0 < J \leq W_k$$

$$\psi_{k}^{(Conv)}(o_{k-1}) = (\sigma(p_{k-1,(I,J)} \mathbf{w}^{(k)} + \mathbf{b}^{(k)})) $$
\todoin{this definition can be corrected/better.}

\subsection{Pooling}
Pooling is a way of reducing the dimensionality of an output. Depending on the task, one may choose from different pooling methods. Similar to convolution operation, pooling methods also work with patches $p_{k-1,(I,J)} \in \mathbb{R}^{K \times K \times m_{k-1}}$ and strides $s$. But this time, instead of applying a weight, bias and activation function, they apply functions. 
The function that we will make use of is $ M : \mathbb{R}^{K \times K \times m} \rightarrow \mathbb{R}^{m}$. By defining different variations of M, we will define \textit{max pooling} and \textit{average pooling}.
\subsubsection{Max Pooling}
Max pooling takes the maximum value in a channel within the patch.
$$ M^{(max)}(p) = \bigg\{max\big(\{p_{i,j,l} \ |\  i \in [1, \ldots, K],  j \in [1, \ldots, K] \}\big) \ |\  l \in [1, \ldots, m] \bigg\} $$
\subsubsection{Average Pooling}
Average pooling averages the values within the patch per channel. 
$$ M^{(avg)}(p) = \{\sum_{i=1}^{K}\sum_{j=1}^{K}\frac{p_{i,j,m}}{K^2} \ | \  l \in [1, \ldots, m] \} $$

\subsection{Deconvolution}


\section{Efficient Operations}
In this section we are going to look at some ways to reduce the computational complexities of fully connected layers and convolutional layers. 

\section{Factorization}
Factorization is approximating a weight matrix using smaller matrices. This has interesting uses with Neural Networks. Assume that we have a fully connected layer $k$. Using factorization, we can approximate $\mathbf{w}^{(k)} \in \mathbb{R}^{m_{k-1} \times m_k}$ using two smaller matrices, $U_{\mathbf{w}^{(k)}} \in \mathbb{R}^{m_{k-1} \times n}$ and $V_{\mathbf{w}^{(k)}} \in \mathbb{R}^{n \times m_{k}}$. If we can find matrices such that $U_{\mathbf{w}^{(k)}}V_{\mathbf{w}^{(k)}} \approx \mathbf{w}^{(k)}$, we can rewrite, 
$$\psi^{(FC)}_k(o) \approx \psi'^{(FC)}_k(o) = \sigma(o^T U_{\mathbf{w}^{(k)}}V_{\mathbf{w}^{(k)}} +\mathbf{b}^{(k)})$$
Therefore, we can reduce the complexity of layer $k$ by setting $n$. As we have mentioned before, $\mathcal{O}(\psi_k^{(FC)}) = \mathcal{O}(m_{k-1}m_k)$. When we approximate this operation, the complexity becomes, 
$$\mathcal{O}(\psi'^{(FC)}_k) = \mathcal{O}(n(m_{k-1}+m_k))$$
Therefore, if there is a good enough approximation, satisfying $n < \frac{m_{k-1}m_k}{m_{k-1}+m_k}$, we can reduce the complexity of a fully connected layer without effecting the results.
One thing that's similar between a convolutional layer and a fully connected layer is that both are performing matrix multiplication to calculate results. The only difference is, a convolutional layer is possibly performing this matrix multiplication many times. Therefore the same technique can be used with convolutional layers. 


\todoin{Luc says: If XW is not exactly equal XOP, how will you deal with this?}
\subsection{SVD}
Singular Value Decomposition (SVD) (\cite{golub1970singular}), is a factorization method that we can use to calculate this approximation. SVD decomposes the weight matrix $\mathbf{w}^{(k)} \in \mathbb{R}^{m_{k-1} \times m_k}$ into $3$ parts. 
$$ \mathbf{w}^{(k)} = USV^T $$
Where, $U \in \mathbb{R}^{m_{k-1} \times m_{k-1}}$ and $V\in \mathbb{R}^{m_{k} \times m_k}$. And $S \in \mathbb{R}^{m_{k-1} \times m_k}$ is a rectangular diagonal matrix. The diagonal values of $S$ are called as the singular values of $M$. Selecting the $n$ highest values from $S$ and corresponding columns and rows from $U$ and $V$, respectively, lets us create a \textit{low-rank decomposition} of $\mathbf{w}^{(k)}$. 
$$ \mathbf{w}^{(k)} \approx U'S'V'^T $$
where $U' \in \mathbb{R}^{m_{k-1} \times n}$, $V' \in \mathbb{R}^{n \times n}$, and $S' \in \mathbb{R}^{n \times m_k}$. By choosing a sufficiently small $n$ value and setting $U_{\mathbf{w}^{(k)}}=U'S'$ and $V_{\mathbf{w}^{(k)}} = V'^T$, we can approximate the weights, and reduce the complexity of a layer. \cite{zhang2016accelerating} applies this method to reduce the execution time of a network by 4 times and increase accuracy by 0.5\%.

\subsection{Weight Sharing}
Introduced by \cite{nowlan1992simplifying}, weight sharing starts with regular weight matrices, $\mathbf{w}^{(k)} \in \mathbb{R}^{m_{k-1} \times m_k}$ where $k \in [1, \ldots, \mathcal{K}]$. Once the weights are learned, they use clustering to find a set of weights, $\mathbf{W} \in \mathbb{R}^{a}$. Then they store the cluster index per weight in $d^{(k)} \in \mathbb{N}^{m_{k-1} \times m_k}$. By redefining $\mathbf{w}^{(k)}_{i,j} = \mathbf{W}_{d^{(k)}_{i,j}}$, they perform weight sharing. Please note that this method does not necessarily reduce model complexity by itself. It reduces the model size by storing indices using less bits. In theory, such a method when applied before factorization should provide a lower rank decomposition. 

\section{Convolution Operation Alternatives}

\subsection{Kernel Composition}
As  \cite{alvarez2016decomposeme} explains, a convolution operation with a weight matrix $\mathbf{w}^{(k)} \in \mathbb{R}^{K \times K \times m_{k-1} \times m_k}$, could be composed using two convolution operations with kernels $\mathbf{w}^{(k,1)} \in \mathbb{R}^{1 \times K \times m_{k-1} \times n}$ and $\mathbf{w}^{(k,2)} \in \mathbb{R}^{K \times 1 \times n \times m_k}$. Their technique, instead of factorizing learned weight matrices, aims to learn the factorized kernels. They also aim to increase non-linearity by adding bias and activation function in between. Therefore defining;

$$ \psi_{(k)}^{(ConvCompose)}(o) = \psi_{(k,2)}^{(Conv)}(\psi_{(k,1)}^{(Conv)}(o))$$

This method forces the separability of the weight matrix as a hard constraint. By performing such an operation, they convert the computational complexity of a convolution operation from $\mathcal{O}(KKm_{k-1}m_k)$ to $\mathcal{O}(Kn(m_{k-1} +m_{k}))$. Suggesting that, if we can find an $n$ satisfying $\frac{\mathcal{O}(KKm_{k-1}m_k)}{\mathcal{O}(Kn(m_{k-1} +m_{k})} > 1$, we can reduce the complexity of this layer. This equation can be rewritten as;
$$ \frac{Km_{k-1}m_k}{m_{k-1} + m_k} > n$$


\subsection{Separable Convolutions}
Separable convolutions separate the standard convolution operation into two parts. These parts are called depthwise convolutions and pointwise convolutions. Depthwise convolution applies a given number of filters on every input channel, one by one therefore results with output channels equal to input channel times number of filters. 
\subsubsection{Depthwise Convolution}
Given a patch $p \in \mathbb{R}^{K \times K \times m}$, depthwise convolution has a weight matrix $\mathbf{w}^{(k, Depthwise)} \in \mathbb{R}^{K \times K \times m}$. For easiness, let's assume variants of $p$ and $\mathbf{w}^{(k, Depthwise)}$, described as $p' \in \mathbb{R}^{K \times K}$ and $\mathbf{w}'^{(k, Depthwise)}  \in \mathbb{R}^{K \times K}$,
$$\mathbf{w}'^{(k, Depthwise)}_m = \{ \mathbf{w}^{(k, Depthwise)}_{i,j,m} \ | \ i \in [1, \ldots, K], j \in [1, \ldots, K] \}$$
$$p'_m = \{ p_{i,j,m} \ | \ i \in [1, \ldots, K], j \in [1, \ldots, K] \}$$
Therefore, depthwise convolution operation $\psi_k^{(Depthwise)} : \mathbb{R}^{W_{k-1} \times H_{k-1} \times m_{k-1}} \rightarrow \mathbb{R}^{W_{k} \times H_{k} \times m_{k-1}} $ and it's complexity can be defined as, 
$$\psi_k^{(Depthwise)}(p) = \{ p'_i\mathbf{w}'^{(k, Depthwise)}_i \ | \ i \in [1, \ldots, m] \} $$
$$ \mathcal{O} (\psi_k^{(Depthwise)}) = \mathcal{O} (H_kW_kK^2m_{k-1}) $$

\subsubsection{Pointwise Convolution}
Pointwise convolution ($\psi^{(Pointwise)}_k : \mathbb{R}^{H_k \times W_k \times m_{k-1}} \rightarrow \mathbb{R}^{H_k \times W_k \times m_{k}}$) is a regular convolution operation with kernel size 1 ($K = 1$). The weight matrix that we'll use for this operation is $\mathbf{w}^{(k, Pointwise)} \in \mathbb{R}^{ m_{k-1} \times m_k}$. As you can see, $\mathbf{w}^{(k, Pointwise)}$ is the same shape as  a fully connected layer weight matrix. The pointwise convolution operation can be defined as,
$$ o_{k-1, (I,J)} \in \mathbb{R}^{1 \times m_{k-1}} \text{ where }  0 < I \leq H_k \text{ and } 0 < J \leq W_k $$
$$ \psi^{(Pointwise)}_k(o_{k-1}) = \{ o_{k-1, (I,J)} \mathbf{w}^{(k, Pointwise)} \ | \ I \in [1, \ldots, H_k], J \in [1, \ldots, W_k]\} $$
$$ \mathcal{O} (\psi_k^{(Pointwise)}) = \mathcal{O} ( H_kW_km_{k-1}m_k)$$
Therefore we can describe $\psi^{(Separable)}_k : \mathbb{R}^{W_{k-1} \times H_{k-1} \times m_{k-1}} \rightarrow \mathbb{R}^{W_{k} \times H_{k} \times m_{k}}$ as,
$$ \psi^{(Separable)}_k(o_{k-1}) = \psi^{(Pointwise)}_k ( \psi^{(Depthwise)}_k(o_{k-1}) ) $$
The complexity of this operation is, 
\begin{equation*}
\begin{split}
	\mathcal{O}(\psi^{(Separable)}_k) =&  \mathcal{O} (\psi^{(Pointwise)}_k) + \mathcal{O} (\psi^{(Depthwise)}_k) \\
							    =& \mathcal{O} (H_kW_km_{k-1}m_k + H_kW_kK^2m_{k-1}) \\
							    =& \mathcal{O} (H_kW_km_{k-1}(m_k + K^2))
\end{split}
 \end{equation*}

\section{Pruning}

Pruning aims to reduce the model complexity by \textit{deleting} the parameters that has low or no impact in the result. \cite{lecun1989optimal} has shown that using second order derivative of a parameter, we can estimate the effect it will have on the training loss. By removing the parameters that have low effect, they have reduced the network complexity and increased accuracy. \cite{Hu:2016aa} has shown that there may be some neurons that are not being activated by the activation function (i.e. ReLU in their case). Therefore, they count the activations in neurons and remove the ones that have are not getting activated. After that they retrain their network and achieve better accuracy than non-pruned network. \cite{han2015learning} shows that we can prune the weights that are very close to 0. By doing that they reduce the number of parameters in some networks about 10 times with no loss in accuracy. To do that, they train the network, prune the unnecessary weights, and train the remaining network again.  \cite{tu2016reducing} shows that using Fisher Information Metric we can determine the importance of a weight. Using this information they prune the unimportant weights. They also use Fisher Information Metric to determine the number of bits to represent a weight. Also, \cite{reed1993pruning} compiled many pruning algorithms.

\subsection{Pruning Weights}
This subcategory of pruning algorithms try to optimize the number of floating point operations by removing some individual values. In theory, it should benefit the computational complexity to remove individual scalars from $\mathbf{w}^{(k)}$. However, we are defining our layer operations using matrix multiplications. To our knowledge, most of the matrix multiplication implementations require dense matrices. Another option would be to use sparse matrix multiplication. But to be able to gain \textit{any} speed up using sparse matrices, we need to prune about 90\% of the weights. Which doesn't seem feasible. 

\subsection{Pruning Nodes}
Since it is not possible to remove individual weights and reduce the computational complexity, we are going to look at another case of pruning. This case focuses on pruning a node and all the weights connected to it. Let's assume two fully connected layers, $k$ and $k+1$. The computational complexity of computing the outputs of these two layers would be $\mathcal{O}(\psi^{(FC)}_{k+1}(\psi^{(FC)}_{k}(o_{k-1})) = \mathcal{O}(m_k(m_{k-1} + m_{k+1})$. Assuming that we have removed a single node from layer $k$, this complexity would drop by $\mathcal{O}(m_{k-1} + m_{k+1})$. 

Similar to the fully connected layer, a convolutional layer $k$ also contains $m_k$ nodes. The only difference is, in a convolutional layer, these nodes are repeated in dimensions $H_k$ and $W_k$. Therefore, it is possible to apply this technique to convolutional layers. 

\iffalse
\subsection{Activation Based Pruning - Fully Connected Layers} \label{sec:activation-based-pruning-convolution}
Activation based pruning, works by looking at individual values in layers, and prunes the layer and corresponding weight row/columns completely. To visualize this, we will assume that the fully connected layers we have defined are, trained to some extent, and activated using ReLU activations. With this definition, if we apply our dataset and count the number of activations in $\mathbf{l_1}$ and $\mathbf{l_2}$, we may realize that there are some neurons that are not being activated at all. By removing these neurons from the layers, we can reduce the number of operations. This removal operation is done by removing neurons based on their activations. 
\todoin{add figure to show what happens when we prune}

\subsection{Activation Based Pruning - Convolution and Deconvolutions}
\todoin{Put references for conv and deconv operations. }
In theory, convolution operation is a matrix multiplication applied on a sliding window. Thus, counting the output feature activations of a convolution operation, we can apply activation based pruning. 

\subsection{Second Order Derivatives (Fischer Information Matrix)}
\fi

\section{Quantization}
A floating point variable can not represent all decimal numbers perfectly. An n-bit floating point variable can only represent $2^{n}$ decimals. The decimals that can not be represented perfectly using 32-bits are going to be represented with some error. Quantization is the process used to represent values using less bits. 

In a higher level, the computational complexity doesn't depend on number of bits. But if we dive deeper in the computer architecture, using less bits to represent variables provide some major advantages. It takes less cpu-cycles to perform an operation, reduces the cost of transferring data from memory to cpu and finally increases the amount of data that can fit into cache. One major disadvantage is, most architectures implement optimizations that speed up 16/32/64-bit floating point operations. By using less bits, we are giving up on these optimizations. 

\section{Improving Network Efficiency}

\subsection{Residual Connections}
\cite{He:2015aa} introduced a method called residual connections. Assuming groups of consecutive layers in out network, we create a residual connection when we add the input of a block to the output of the block to calculate the input of the next block. Let's assume a block $b$ with input $o_{b-1} \in \mathbb{R}^{m_{b-1}}$ and output  $o_{b} \in \mathbb{R}^{m_{b}}$. We call these two blocks residually connected if we perform $o'_b = o_b + o_{b-1}$ and set $o'_b$ as the input of the next block. 

Residual connections allow us to train deeper networks by preventing the vanishing gradient problem. As we increase the number of layers in a neural network, the gradient values for weights in the former layers start getting smaller and smaller. They get so small that they become irrelevant and don't change anything. Residual connections increase the effect of deeper layers for calculating the output. Therefore, their gradients stay do not vanish because they have significant contribution to the result.

\subsection{Batch Normalization}
\cite{ioffe2015batch} introduced a method called batch normalization. Batch normalization aims to normalize the output distribution of a every node in a layer. By doing so it allows the network to be more stable. 

Assume the layer k with $o_k \in \mathbb{R}^{m_k}$ where $m_k$ is the number of nodes. Batch normalization has four parameters. Mean is $\mu_k \in \mathbb{R}^{m_k}$, variance is $\sigma_k \in \mathbb{R}^{m_k}$, scale is $\gamma_k \in \mathbb{R}^{m_k}$ and offset is $\beta_k \in \mathbb{R}^{m_k}$. 

Since we are interested in normalizing the nodes, even if $k$ was a convolutional layer, the shape of these parameters would not change.
$$ BN(o_k) = \frac{\gamma_k(o_k-\mu_k)}{\sigma_k}+\beta_k $$

\subsection{Regularization}
Regularization methods aim to prevent overfitting in neural networks. Overfitting is the case where the weights of a a neural network converge for the training dataset. Meaning that the network performs very very good for the training dataset, while it is not generalized to work with any other data. Regularization methods try to prevent this.

One common regularization method is to add a new term to the loss, which punishes high weight values. We also add a term $\lambda$ which determines the effect of this regularization. This parameter, if too high would prevent the network from learning, if too low would have no effect. 

\subsubsection{L1 Regularization}
$$ L1 = \lambda \sum_{w \in \mathbf{W}} |w| $$

\subsubsection{L2 Regularization}
$$ L2 = \lambda \sum_{w \in \mathbf{W}} w^2 $$

\section{Efficient Structures}
Some structures help neural networks represent more information using less parameters. We are going to look at some structures that are known to work well with convolutional neural networks.

\iffalse
\todoin{I am not adding this because we are not using Inception blocks anywhere in this thesis.}
\subsection{Inception Blocks}
An inception block (\cite{Szegedy:2014aa}, \cite{Szegedy_2016_CVPR}, \cite{DBLP:journals/corr/SzegedyIV16}) is a combination of various layers. The inception block $k$ gets an input $o_{k-1}$ then applies multiple convolutional (or pooling) layers to this input. The results of these layers are combined to produce the output of the inception block. There are different combinations of inception blocks. 
\fi

\subsection{Residual Blocks}
\cite{He:2015aa} introduced two types of residually connected blocks. First is called a residual block, consisting of two convolution operations and a residual connection between the input and the output of the block. They have trained networks with and without residual connections on ImageNet dataset. Their results show that introduction of the residual connections reduce the top-1 error rate of the 34 layer network from $\%28.54$ to $\%25.3$.
\todoin{explain this as you explained residual bottleneck blocks.}

\subsection{Residual Bottleneck Blocks}
\cite{He:2015aa} have used residual blocks to train networks up to 34 layers. For networks greater than 34 layers, they have introduced a second block called as residual bottleneck block. Residual bottleneck blocks consist of three convolution operations with different kernel sizes and number of output nodes. 

Before explaining how residual bottleneck blocks are configured, we need more notations to define the layers inside the same block. Let's assume the block $b$ with input $o_{b-1} \in \mathbb{R}^{H_{b-1} \times W_{b-1} \times m_{b-1}}$. We will index the layers inside the block $b$ with a pair $(b, k)$ where $k$ stands for the index of convolutional layer inside the block. For example if we're talking about the second convolutional layer in block $b$, the input of this convolutional layer would be $o_{(b, 1)} \in \mathbb{R}^{H_{(b, 1)} \times W_{(b, 1)} \times m_{(b, 1)}}$ and the output would be $o_{(b, 2)} \in \mathbb{R}^{H_{(b, 2)} \times W_{(b, 2)} \times m_{(b, 2)}}$. also since we are talking about layers with different kernel sizes, we need to define them with indexes as well. Therefore the kernel size of convolutional layer $(b,k)$ would be $K_{(b,k)} \in \mathbb{R}$.
Therefore, we can define the residual bottleneck block as;
$$ \psi_b^{(ResidualBottleneck)}: \mathbb{R}^{H_{b-1} \times W_{b-1} \times m_{b-1}} \rightarrow  \mathbb{R}^{H_{b} \times W_{b} \times m_{b}} $$
with the helper function $S$
$$ \psi_b^{(ResidualBottleneck)}(o) =  S(o) + \psi_{(b, 3)}^{(Conv)}(\psi_{(b, 2)}^{(Conv)}(\psi_{(b, 1)}^{(Conv)}(o))) $$
\begin{equation*}
\label{eq:output_of_layers}
    S_b(o) = 
\begin{cases}
    o, &\text{if } (W_b, H_b, m_b) = (W_{b-1}, H_{b-1}, m_{b-1})\\
    P_b(M_b^{(avg)}(o)),& \text{otherwise}\\
\end{cases}
\end{equation*}
where $P$ is a padding function that equalizes the number of nodes in the outputs and $M_b^{(avg)}$ is the average pooling function that equalizes the width and height dimensions of the input and output of the block.
To their definition the first convolution operation in the block reduces the number of nodes with kernel size $K_{(b,1)}=1$. The number of nodes is defined with a dependency to the stride of block ($s_b \in \{1,2\})$ as $m_{(b,1)} = m_{b}/(4/s_b)$. By that logic, if a residual block is reducing the width and height by half, it is doubling the number of nodes to represent more information. Second convolution operation kernel size $K_{(b,2)}=3$ and has the same number of nodes as the previous layer $m_{(b,2)} = m_{(b,1)}$. The third convolutional layer quadruples the number of nodes ($m_{(b,3)} = 4m_{(b,1)}$) with kernel size $K_{(b,3)} = 1$.

Their 50-layer network using residual bottleneck blocks achieves $\%22.85$ top-1 error rate on ImageNet dataset.


\iffalse
\begin{table}[]
\centering
\begin{tabular}{ | c | c | c | c | }
\hline
layer name			& output size 					& 34-layer																& 50-layer																			\\ \hline
input image			& $224 \times 224$				& \multicolumn{2}{c|}{}																																	\\ \hline
conv1				& $112 \times112$				& \multicolumn{2}{c|}{$ 7 \times 7$, $64$, stride $2$}																												\\ \hline
\multirow{2}{*}{conv2\_x}	& \multirow{2}{*}{$56 \times 56$} 	& \multicolumn{2}{c|}{$3 \times 3$ max pool, stride $2$}																											\\ \cline{3-4} 
					&							& $\begin{bmatrix} 3 \times 3, &   64 \\ 3 \times 3, &   64 \end{bmatrix} \times 3 $		& $\begin{bmatrix}1 \times 1, & 64 \\ 3 \times 3, & 64 \\ 1 \times 1, & 256 \end{bmatrix}^{} \times 3 $ 		\\ \hline
conv3\_x				& $28 \times 28$				& $\begin{bmatrix} 3 \times 3, & 128 \\ 3 \times 3, & 128 \end{bmatrix} \times 3 $		& $\begin{bmatrix}1 \times 1, & 128 \\ 3 \times 3, & 128 \\ 1 \times 1, & 512 \end{bmatrix} \times 3$		\\ \hline
conv4\_x				& $14 \times 14$				& $\begin{bmatrix} 3 \times 3, & 256 \\ 3 \times 3, & 256 \end{bmatrix} \times 3 $		& $\begin{bmatrix}1 \times 1, & 256 \\ 3 \times 3, & 256 \\ 1 \times 1, & 1024 \end{bmatrix} \times 3$		\\ \hline
conv5\_x				& $  7 \times   7$				& $\begin{bmatrix} 3 \times 3, & 512 \\ 3 \times 3, & 512 \end{bmatrix} \times 3 $		& $\begin{bmatrix}1 \times 1, & 512 \\ 3 \times 3, & 512 \\ 1 \times 1, & 2048 \end{bmatrix} \times 3$		\\ \hline
					& $  1 \times   1$				&\multicolumn{2}{c|}{average pool, 1000-d fc, softmax}																											\\ \hline
\multicolumn{2}{| c |}{FLOPs}							& $3.6 \times 10^9$														& $3.8 \times 10^9$																	\\ \hline
\multicolumn{2}{| c |}{top-1 error ($\%$)}						& $21.53$																& $20.74$																			\\ \hline
\multicolumn{2}{| c |}{top-5 error ($\%$)}						& $5.60$																& $5.25$																			\\ \hline
\multicolumn{2}{| c |}{top-1 error \small{($\%$, \textbf{10-crop} testing)}}						& $24.19$																& $22.85$																			\\ \hline
\multicolumn{2}{| c |}{top-5 error \small{($\%$, \textbf{10-crop} testing)}}						& $7.40$																& $6.71$																			\\ \hline
\end{tabular}
\caption{Comparison of bottleneck blocks (50-layer) with stacked $ 3 \times 3$ layers (34-layer). }
\label{tab:bottleneck-comparison}
\end{table}
\fi

\section{Datasets}
\subsection{MNIST}
MNIST dataset \cite{lecun1998mnist} consists of 60.000 training and 10.000 test samples. Each sample is a $28 \times 28$ black and white image of a handwritten digit ($0$ to $9$). To our knowledge, best model trained on MNIST achieve almost zero ($\%0.23$, \cite{DBLP:journals/corr/abs-1202-2745}) error rate. 
\subsection{CIFAR10}
CIFAR10 dataset \cite{krizhevsky2009learning} consists of 50.000 training and 10.000 test samples. Each sample is a $32 \times 32$ colored image belonging to one of 10 classes. The classes are airplane, automobile, bird, cat, deer, dog, frog, horse, ship and truck. To our knowledge, best models trained on CIFAR10 achieve $\%3.47$ (\cite{DBLP:journals/corr/Graham14a}) error rate.

\section{Tools}
\subsection{Tensorflow}
To develop and train our neural networks, we are going to be using tensorflow \cite{abadi2016tensorflow}. Tensorflow provides us the necessary tools to deploy our trained models on mobile devices. 

\chapter{Methods}
% !TEX root = ../thesis.tex
\section{Dependencies}
\subsection{Tensorflow}
In our research, we will strictly use Tensorflow \cite{abadi2016tensorflow}. \todo[inline]{What is tensorflow, why we chose it, what are the advantages of using it, what are the limitations that come with it}

\section{Pruning}
Pruning aims to reduce the number of operations by deleting the parameters that has low or no impact in the result. Studies show that applying this method in an ANN is effective in reducing the model complexity, improving generalization, and they are effective in reducing the required training cycles. In our experiments we will try to reproduce these effects.
To visualize these methods, let's think of two fully connected layers, $\mathbf{l_1}$ and $\mathbf{l_2}$. $\mathbf{l_1}$ is the input of this operation and it consists of $N$ values, $\mathbf{l_1}=(l_{11}, l_{12}, ..., l_{1N})$. $\mathbf{l_2}$ is the output of this operation consists of $M$ values, $\mathbf{l_2}=(l_{21}, l_{22}, ..., l_{2M})$. Between these two layers, there is a weight matrix $W$ with size $N \times M$. The operation, that we want to optimize is, $\mathbf{l_2} = \mathbf{l_1}W$.
\todo[inline]{maybe explain in more detail and give examples of pruning algorithms here. (e.g. Optimal Brain Damage, Second order derivatives for network pruning: Optimal Brain Surgeon, Optimal Brain Surgeon and general network pruning, SEE Pruning Algorithms-a survey from R. Reed)}

\subsection{Pruning Individual Weights}
With this subcategory of pruning algorithms we want to optimize the number of floating point operations by removing some values from $W$. Theoretically, it makes sense to remove individual scalars from W, and exclude operations related to them. This would ideally reduce the required number of floating point operations. But in our library of our choice, Tensorflow, matrix multiplication implementation \texttt{tf.matmul} do not consider such a change. It takes two fixed size matrices, and does the computations using all of their values. Tensorflow also has another matrix multiplication operation, \texttt{tf.sparse\_tensor\_dense\_matmul}. This operation takes a sparse matrix and a dense matrix as inputs and outputs a dense matrix. To implement this method, we could convert W to a sparse tensor after pruning the weights. But, Tensorflow documentations about this method state;
\begin{itemize}
\item Will the SparseTensor A fit in memory if densified?
\item Is the column count of the product large ($>> 1$)?
\item Is the density of A larger than approximately $15\%$?
\end{itemize}
"If the answer to several of these questions is yes, consider converting the SparseTensor to a dense one.". In our terms, SparseTensor A is corresponding to the pruned version of $W$. 

Since $W$ was already dense before, we can assume that the answer to the first question is yes. The column count of our product is $M$ which is much larger than $1$ in some cases. Also we don't know anything about the density of pruned version of $W$. Looking at these facts, we are assuming that implementing this operation will be problematic. Instead of delving deeper into these problems to evaluate this method, we will move on to other methods.

\subsection{Activation Based Pruning}
Activation based pruning, works by looking at individual values in layers, and prunes the layer and corresponding weight row/columns completely. To visualize this, we will assume that the fully connected layers we have defined are, trained to some extent, and activated using ReLU activations. With this definition, if we apply our dataset and count the number of activations in $\mathbf{l_1}$ and $\mathbf{l_2}$, we may realize that there are some neurons that are not being activated at all. By removing these neurons from the layers, we can reduce the number of operations to some extent. This removal operation is done by removing the non-activated neurons from $\mathbf{l_1}$ and $\mathbf{l_2}$, followed by removing corresponding columns and rows from $W$.



\chapter{Results}
% !TEX root = ../thesis.tex
\iffalse
Results

    The results are actual statements of observations, including statistics, tables and graphs.
    Indicate information on range of variation.
    Mention negative results as well as positive. Do not interpret results - save that for the discussion. 
    Lay out the case as for a jury. Present sufficient details so that others can draw their own inferences and construct their own explanations. 
    Use S.I. units (m, s, kg, W, etc.) throughout the thesis. 
    Break up your results into logical segments by using subheadings
    Key results should be stated in clear sentences at the beginning of paragraphs.  It is far better to say "X had significant positive relationship with Y (linear regression p<0.01, r^2=0.79)" then to start with a less informative like "There is a significant relationship between X and Y".  Describe the nature of the findings; do not just tell the reader whether or not they are significant.  
\fi


\section{Pruning}
In this section we consider the configurations that have pruned the most nodes as our results. Since the combination of variables and parameters that we have used are very big, we chose not to report the configurations that are not working. 

As we have explained in Chapter~\ref{cha:methods}, we have pruned the nodes of a fully connected neural network trained to predict the summation of two floating point values, and an autoencoder trained on the MNIST dataset. We ran the experiments at least three times with each configuration to verify our results.

\subsection{Fully Connected Networks}
We have initialized and run training cycles on the fully connected network. By applying distortions to remaining weights between training cycles, we have achieved the optimum result we have shown in Figure~\ref{fig:optimum_fc_summation}, with only one node in the hidden layer. 

When distortions were applied between training cycles, both pruning criteria worked equally good. For both criteria, we were able to achieve the optimum network structure using a fixed threshold of $0$. In other words, for activation count criteria, we have pruned the nodes that were not being activated, and for activation variance criteria, we have pruned the nodes that had zero variance in values.

The loss was almost zero for the training and test datasets. Even though we have achieved the optimal shape, the model was overfitting for the mean and standard deviation parameters we set for the random number generator while generating the training dataset.

In the experiments that we did not apply distortions we could not find a configuration that achieves the optimal network structure. In most of the configurations we were unable to find any nodes to prune after the first training cycle. We could not see any difference between different regularization terms or pruning criteria.

\subsection{Convolutional Neural Networks}
In this setting using activation variance criteria an L2 regularization, we have achieved the most optimum result. We did not see any improvement by using distortions. Using L2 regularization, compared to no regularization and L1 regularization, we have seen an improvement in the number of nodes pruned in every training cycle and the final result. 

We were unable to find a good threshold for activation count criteria. However, doing a basic outlier selection with the activation variances worked best. Given that $v$ is the variance vector each output feature of a layer, by choosing a lower and upper boundaries as,
$$ mean(v) - 2*var(v) < x < mean(v) + 2*var(v) $$
and removing the nodes that are not outside these boundaries, we were able to find the most optimum solutions. 

The most ideal case was with no distortion, l2 regularization (with $\lambda = 0.01$) and pruning nodes based on activation variance criteria given above. Using this setting, we have pruned the autoencoder from $1-32-64-32-1$ to $1-2-4-3-1$ nodes per layer, which is a much better result compared to the baseline we have defined. It took us $10\pm4$ training cycles to achieve this result. The loss we have achieved is very close to 0. The encoder we have achieved is shown in Figure~\ref{fig:pruned_autoencoder}.
\begin{figure}[!h]
    \begin{subfigure}{1\textwidth}
        \hspace{-.1\linewidth}
        \includegraphics[width=1.2\linewidth]{images/over_parameterized_autoencoder.pdf}
        \caption{Initial autoencoder configuration.}
        \label{fig:initial_autoencoder}
    \end{subfigure}
    \begin{subfigure}{1\textwidth}
        \hspace{-.1\linewidth}
        \includegraphics[width=1.2\linewidth]{images/optimum_autoencoder.pdf}
        \caption{Resulting autoencoder configuration.}
        \label{fig:pruned_autoencoder}
    \end{subfigure}
    \caption{Pruned autoencoder compared to initial autoencoder.}
    \label{fig:pruned_autoencoder}
\end{figure}

\newpage
\section{Convolution Operation Alternatives}
We ran experiments to see which operation is a cheaper alternative to convolution operation. We ran our experiments 10 times to validate our results.
\subsection{MNIST}
In our experiments with MNIST dataset, we have not seen a comparable difference between experiments with different operations. All of the experiments have resulted with $99\pm0.3\%$ top-1 accuracy, with no clearly visible difference.
\subsection{CIFAR-10}
The results of our experiments in CIFAR-10 dataset are given in Table~\ref{tab:convolution_alternative_results}.
\begin{table}[]
\centering
\begin{tabular}{| l | r | r | }
\thead{Model}                                                                     & \thead{Mean Accuracy} & \thead{Max Accuracy} \\ \hline
Convolution (Baseline)                                                             & 81.84                  & 82.56                 \\ \hline
Kernel Composing Conv.                                                             & 81.98                  & 82.51                 \\ \hline
Separable Convolution                                                              & 82.11                  & 82.53                 \\ \hline
\begin{tabular}[c]{@{}l@{}}Separable Convolution\\ with non-linearity\end{tabular} & \textbf{82.16}         & \textbf{82.75}        \\ \hline
\end{tabular}
\caption{Mean and max of top-1 accuracy results of 10 runs, using CIFAR-10 validation dataset.}
\label{tab:convolution_alternative_results}
\end{table}
As emphasized in the table, separable convolution operation with non-linearity performed slightly better than the rest of the operations.

\section{Small Models}
In this section we will present the results of our experiments on small models. We show how small models perform compared to large ones and see how pruning and approximation methods work with them.
\subsection{Models}
\label{sec:result_models}
\subsubsection{CIFAR-10}
The model we have defined for this task (see Figure~\ref{fig:separable_resnet_cifar10}), has achieved a maximum of 91.1\% top-1 classification accuracy on CIFAR-10 test dataset in 10 training sessions. As we have shown in Table~\ref{tab:cifar-model-vs-resnet-20}, compared to ResNet-20 (\cite{He:2015aa}), our model has performed slightly worse in terms of top-1 classification accuracy. However, in terms of floating point operations and number of parameters, our model is about 2 times smaller in terms of number of parameters, and requires 4 times less floating point operations to perform an inference. 

\begin{table}[!h]
\hspace{-12px}
\begin{tabular}{|l|r|r|r|r|r|}
\thead{Model} & \thead{\# params} & \thead{\# layers} & \thead{\# blocks} & \thead{\# ops} & \thead{top-1 ac. (\%)} \\ \hline
Our Model                          & $\mathbf{0.12}$               & 27  & 6                & $\mathbf{\sim 13M}$             & 91.10                        \\ \hline
ResNet-20                          & 0.27               & 20 & 9                 & $\sim$ 40M             & $\mathbf{91.25}$                        \\ \hline
\end{tabular}
\caption{Our small model compared with ResNet-20 from \cite{He:2015aa}.}
\label{tab:cifar-model-vs-resnet-20}
\end{table}

\subsubsection{ImageNet}
The model we have defined for this task (see Figure~\ref{fig:model}), has achieved a maximum of 63 \% top-1 classification accuracy on ImageNet test dataset. We trained this model only once. Compared to ResNet-34 (\cite{He:2015aa}), our model has performed very poorly.

\subsection{Pruning Small Models}
\label{sec:pruning_small_models}
Using the pruning criteria we have defined for autoencoders, we have tried to prune the CIFAR-10 model. However, we were unable to prune a significant amount of nodes and recover the accuracy in the next training cycle.

\subsection{Approximating Small Models}
We ran the approximation tool we have defined on our best model (91.1 \% top-1 accuracy) from our CIFAR-10 experiments. Using various pruning thresholds and error thresholds, we could not find any factorization that would make this model faster while preserving the accuracy.

\subsection{Quantization}
We have quantized the CIFAR-10 model, converting 32-bit floating point operations to 8-bit floating point operations. However, in our benchmarks, we have seen that this method has slowed down the inference speed by almost half. We haven't seen any significant change in the model accuracy.

\iffalse
\section{Benchmarks and Comparisons}
We have benchmarked 20 models in different sizes and shapes using our benchmarking device. Here we will report the models that we found important.

Inception-Resnet-v2 (\cite{DBLP:journals/corr/SzegedyIV16}) has a model size of 224 MB and our benchmarking device could execute 0.3-0.4 inferences per second. The cpu-utilization with this model was 0.006568. On a 4 core device, this number is extremely low, so we can conclude this model is cursed by the memory bottleneck bandwidth. 

We have seen that ResNet-50 (\cite{He:2015aa}, \cite{he2016identity}) with a model size of 100 MB got executed for 1.3-1.7 inferences per second with a cpu-utilization of 2.7. We found this amount to be great compared to the previous model.  

Among many models we have inspected, only 1.0 MobileNet-224 (\cite{howard2017mobilenets}) performed faster than our separable resnet. 1.0 Mobilenet-224 had a model size of 17.1 MB and our benchmarking device could perform 5 to 6 inferences per second with a cpu-utilization of 2.7. 

\begin{table}[!h]
\hspace{-30px}
\label{my-label}
\begin{tabular}{|l|r|r|r|r|r|}
\thead{\small Model}      & \thead{\small top-1 ac.(\%)} & \thead{\small \# layers}                                                          & \thead{\small model size} & \thead{\small cpu util.} & \thead{\small inf./sec.} \\ \hline
Inception-resnet-v2 & 80.1 & \textbf{131}                                                                                                  & 224 MB                      & \textbf{0.006}                    & 0.35                      \\ \hline
ResNet-50           & 75.9\footnote{\url{https://github.com/facebook/fb.resnet.torch/blob/master/pretrained/README.md}}  & 50 & 100 MB                      & 2.7                      & 1.5                       \\ \hline
1.0 Mobilenet-224   & 70.6        &    28                                                                                       & 17.1 MB                     & 2.7                      & \textbf{5.5}                       \\ \hline
Our ImageNet Model  & 63         & 77                                                                                           & 12.1 MB                      & 3.1                      & 4.5                      \\ \hline
\end{tabular}
\caption{Benchmark results for various models.}
\end{table}
\fi


\chapter{Discussion}
% !TEX root = ../thesis.tex

\iffalse
 Start with a few sentences that summarize the most important results. The discussion section should be a brief essay in itself, answering the following questions and caveats: 

    What are the major patterns in the observations? (Refer to spatial and temporal variations.)
    What are the relationships, trends and generalizations among the results?
    What are the exceptions to these patterns or generalizations?
    What are the likely causes (mechanisms) underlying these patterns resulting predictions?
    Is there agreement or disagreement with previous work?
    Interpret results in terms of background laid out in the introduction - what is the relationship of the present results to the original question?
    What is the implication of the present results for other unanswered questions in earth sciences, ecology, environmental policy, etc....?
    Multiple hypotheses: There are usually several possible explanations for results. Be careful to consider all of these rather than simply pushing your favorite one. If you can eliminate all but one, that is great, but often that is not possible with the data in hand. In that case you should give even treatment to the remaining possibilities, and try to indicate ways in which future work may lead to their discrimination.
    Avoid bandwagons: A special case of the above. Avoid jumping a currently fashionable point of view unless your results really do strongly support them. 
    What are the things we now know or understand that we did not know or understand before the present work?
    Include the evidence or line of reasoning supporting each interpretation.
    What is the significance of the present results: why should we care? 

This section should be rich in references to similar work and background needed to interpret results. However, interpretation/discussion section(s) are often too long and verbose. Is there material that does not contribute to one of the elements listed above? If so, this may be material that you will want to consider deleting or moving. Break up the section into logical segments by using subheads. 
\fi

\section{Pruning and Factorization}
In our experiments with pruning, we started with very large models for the given problem. We believe that this might have left a false impression. As it did in our experiments, pruning a model would not reduce the model complexity by 1000 times for every model. Depending on the model size and the problem definition, pruning will help us reduce the model complexity, but if our model is sufficiently small for the problem definition, it may not help us at all. In our experiments we have seen that pruning separable resnet did not reduce the model complexity significantly. 

Similarly, if we apply factorization on a very large model, we gain huge speed ups with minor effort. But if our model is compact enough, the complexity gains from these models reduce significantly.

\section{Using Tensorflow}
We have been using latest versions of Tensorflow. It comes with some advantages, such as:
\begin{itemize}
\item We do not implement lower level operations (such as convolutions). It gives us the opportunity to focus on higher level implementations, such as pruning, or factorization. 
\item Most of the operations are highly optimized for many platforms and devices. If we were to implement a model in C++, we'd have to implement it twice, one for training in GPU and another for running in the mobile device. 
\item Tensorflow provides the necessary tools to deploy models on mobile devices.
\end{itemize}

And it comes with some disadvantages, such as:

\begin{itemize}
\item When we started our work, Tensorflow was in version 0.10. By the date we write this, it is on 1.2. There have been 4 major releases that we had to modify our codebase for.
\item Not all operations are properly implemented. For example, before version 1.2, Tensorflow implementation of separable convolutions were not very well optimized. They were as fast as convolution operations. Before that we could only hope that they would optimize their implementation.
\item It is difficult to implement operations (e.g. ef operator \cite{afrasiyabi2017energy}) or play around with existing ones. The documentation describing C++ internals and build procedures (as of Tensorflow 1.2) are not good enough. 
\item Tensorflow does not provide tools to implement low-bit variables (e.g. a 2-bit integer). So it is not possible to implement some methods that make use of variable width decimals. This limitation makes some methods impossible to use or useless. For example it is not possible to use methods that represent weights using variable width decimals. Also, storing low bit weight indices in combination with a small global weight array to reduce the model size is useless. Since we can not use low bit integers to represent these indices, our model size does not shrink at all.
\end{itemize}

\section{Operation Comparison}
In Section \ref{sec:conv_alternatives}, we have defined a neural network to compare convolution, separable convolution and kernel composing convolution operations. Before our experiments, we have tried to find the best settings for some parameters, such as learning rate, regularization constant and optimizer. We think that using the same settings may have influenced our results. Especially because the number of parameters change considerably when we use separable convolutions, instead of convolutions. 

\iffalse
\section{Working with ImageNet}
Most of the research compares models based on ImageNet\cite{howard2017mobilenets} 
\fi

\section{Model Comparison}
Comparing models that aim for mobile devices is difficult. First, there are device/chip specific properties (i.e. l1-cache size and bandwidth speed) that in theory effect the speed of the model greatly. In theory, mobile device performance of a model depends on the amount of floating point multiplications and the model size. The amount of floating point multiplications would modify the model speed almost linearly. The model size is important, because in theory, it determines the amount of data transferred from memory to l1-cache. If our model and the required space to perform the operations in it are sufficiently small to fit in the l1-cache, it would speed up our model greatly by getting rid of all cache-misses. If these were bigger than the l1-cache, it would create a lack of storing and loading from memory, which would lead to waiting for data coming from memory (memory bandwidth bottleneck). So in theory, model size would effect the performance non-linearly. In practice, we could not find a good way of predicting the number of inferences based on number of floating point operations and model size. 

One thing that greatly affects this process is the implementation of the model executor. We think that the Tensorflow implementation of the model executor is not meant to load the whole model on the cpu. 

\iffalse
But as we understood, Tensorflow implementation is not executing models on model level, but on layer level (based on variable scopes). We think that is the case because when we look at a sufficiently small model's benchmarking results, we saw that the amount of context switches were equal to the number of variable scopes times the number of inferences.
\fi

\chapter{Conclusion}
% !TEX root = ../thesis.tex

\iffalse
    What is the strongest and most important statement that you can make from your observations? 
    If you met the reader at a meeting six months from now, what do you want them to remember about your paper? 
    Refer back to problem posed, and describe the conclusions that you reached from carrying out this investigation, summarize new observations, new interpretations, and new insights that have resulted from the present work.
    Include the broader implications of your results. 
    Do not repeat word for word the abstract, introduction or discussion.
\fi

In this research, we have investigated some methods to reduce the computational cost of convolutional neural networks. To do that, we experimented with some methods that could be used to define models with lower computational cost. We also experimented with some methods to reduce the computational complexity of a given model. 

To be able to experiment with pruning using larger models, we have implemented a tool to describe pruning routines. We have also implemented a tool that applies simple quantization, pruning and factorization methods to trained models. Using these tools, we have observed that these methods reduce the computational cost of sufficiently large models.

In our experiments we have observed that the models using separable convolutions with non-linearity results with a slightly better accuracy compared to models using convolution or kernel compositing convolution operations while requiring a significantly smaller number of operations. Using them, we have redefined residual blocks and designed a model that achieves similar results to ResNet-20 on CIFAR-10 classification task. Our model is two times wider, however it has fewer residual blocks, using two times fewer parameters and requiring 3 times fewer operations. However, more work needs to be done to achieve similar results using ImageNet dataset.

When developing models aimed for processing power restricted environments, we think that designing and training small models based on the requirements is a more stable alternative to compressing large networks. We have seen that wider and shallower residual networks using separable residual blocks are one way of designing such models.


\bibliographystyle{alpha}
\bibliography{references}

\end{document}
