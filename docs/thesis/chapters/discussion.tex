% !TEX root = ../thesis.tex

\iffalse
 Start with a few sentences that summarize the most important results. The discussion section should be a brief essay in itself, answering the following questions and caveats: 

    What are the major patterns in the observations? (Refer to spatial and temporal variations.)
    What are the relationships, trends and generalizations among the results?
    What are the exceptions to these patterns or generalizations?
    What are the likely causes (mechanisms) underlying these patterns resulting predictions?
    Is there agreement or disagreement with previous work?
    Interpret results in terms of background laid out in the introduction - what is the relationship of the present results to the original question?
    What is the implication of the present results for other unanswered questions in earth sciences, ecology, environmental policy, etc....?
    Multiple hypotheses: There are usually several possible explanations for results. Be careful to consider all of these rather than simply pushing your favorite one. If you can eliminate all but one, that is great, but often that is not possible with the data in hand. In that case you should give even treatment to the remaining possibilities, and try to indicate ways in which future work may lead to their discrimination.
    Avoid bandwagons: A special case of the above. Avoid jumping a currently fashionable point of view unless your results really do strongly support them. 
    What are the things we now know or understand that we didn't know or understand before the present work?
    Include the evidence or line of reasoning supporting each interpretation.
    What is the significance of the present results: why should we care? 

This section should be rich in references to similar work and background needed to interpret results. However, interpretation/discussion section(s) are often too long and verbose. Is there material that does not contribute to one of the elements listed above? If so, this may be material that you will want to consider deleting or moving. Break up the section into logical segments by using subheads. 
\fi


\iffalse

what we couldn't do goes here

Weight Sharing
To be able to implement such a representation, we can use 2 potential functions of Tensorflow. One is \texttt{tf.gather} which selects the given indices from a given matrix. The other is \texttt{tf.embedding\_lookup} which works with a bucket of values and returns the given keys/indices from the given bucket. However, both methods accept the indices as a matrix of 32-bit or 64-bit integers. Therefore storing indices instead of weights would not yield with any improvements in the model size. Without an implementation of these methods using low-bit integers, it is not possible to exploit their usefulness.

\fi