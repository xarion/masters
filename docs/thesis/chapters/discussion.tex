% !TEX root = ../thesis.tex

\iffalse
 Start with a few sentences that summarize the most important results. The discussion section should be a brief essay in itself, answering the following questions and caveats: 

    What are the major patterns in the observations? (Refer to spatial and temporal variations.)
    What are the relationships, trends and generalizations among the results?
    What are the exceptions to these patterns or generalizations?
    What are the likely causes (mechanisms) underlying these patterns resulting predictions?
    Is there agreement or disagreement with previous work?
    Interpret results in terms of background laid out in the introduction - what is the relationship of the present results to the original question?
    What is the implication of the present results for other unanswered questions in earth sciences, ecology, environmental policy, etc....?
    Multiple hypotheses: There are usually several possible explanations for results. Be careful to consider all of these rather than simply pushing your favorite one. If you can eliminate all but one, that is great, but often that is not possible with the data in hand. In that case you should give even treatment to the remaining possibilities, and try to indicate ways in which future work may lead to their discrimination.
    Avoid bandwagons: A special case of the above. Avoid jumping a currently fashionable point of view unless your results really do strongly support them. 
    What are the things we now know or understand that we didn't know or understand before the present work?
    Include the evidence or line of reasoning supporting each interpretation.
    What is the significance of the present results: why should we care? 

This section should be rich in references to similar work and background needed to interpret results. However, interpretation/discussion section(s) are often too long and verbose. Is there material that does not contribute to one of the elements listed above? If so, this may be material that you will want to consider deleting or moving. Break up the section into logical segments by using subheads. 
\fi

\section{Pruning and Factorization}
In our experiments with pruning, we started with very large models for the given problem. We believe that this might have left a false impression. As it did in our experiments, pruning a model would not reduce the model complexity by 1000 times for every model. Depending on the model size and the problem definition, pruning will help us reduce the model complexity, but if our model is sufficiently small for the problem definition, it may not help us at all. In our experiments we have seen that pruning separable resnet did not reduce the model complexity significantly. 

Similarly, if we apply factorization on a very large model, we gain huge speed ups with minor effort. But if our model is compact enough, the complexity gains from these models reduce significantly.

\section{Using Tensorflow}
We have been using latest versions of Tensorflow. It comes with some advantages, such as:
\begin{itemize}
\item We don't worry about implementing lower level operations.
\item Most of the operations are highly optimized form any platforms and devices. If we were to implement a model in C++, we'd have to implement it twice, one for training in GPU and another for running in the mobile device. 
\item Tensorflow provides the necessary tools to deploy models on mobile devices.
\end{itemize}

And it comes with some disadvantages, such as:

\begin{itemize}
\item When we started our work, tensorflow was in version 0.10. Now it is on 1.2. There have been 4 major releases that we had to modify our codebase for.
\item Not all operations are properly implemented. For example, before version 1.2, Tensorflow implementation of separable convolutions were not very well optimized. They were as fast as convolution operations. Before that we could only hope that they would optimize their implementation.
\item It is extremely difficult to implement operations (e.g. ef operator \cite{afrasiyabi2017energy}) or play around with existing ones because (as of Tensorflow 1.2) there exists no documentation describing C++ internals and build procedures. Since we chose not to go into i
\item Tensorflow doesn't provide tools to implement low-bit variables (e.g. a 2-bit integer). So it is not possible to implement some methods that make use of variable width decimals. This limitation makes some methods impossible to use or useless. For example it is not possible to use methods that represent weights using variable width decimals. Also, storing low bit weight indices in combination with a small global weight array to reduce the model size is useless. Since we can't use low bit integers to represent these indices, our model size doesn't shrink at all.
\end{itemize}

\iffalse
\section{Working with ImageNet}
Most of the research compares models based on ImageNet\cite{howard2017mobilenets} 
\fi
\section{Model Comparison}
Comparing models that aim for mobile devices is hard. First, there are device/chip specific properties (i.e. l1-cache size and bandwidth speed) that in theory effect the speed of the model greatly. In theory, mobile device performance of a model depends on the amount of floating point multiplications and the model size. The amount of floating point multiplications would modify the model speed almost linearly. The model size is important, because in theory, it determines the amount of data transferred from memory to l1-cache. If our model and the required space to perform the operations in it are sufficiently small to fit in the l1-cache, it would speed up our model greatly by getting rid of all cache-misses. If these were bigger than the l1-cache, it would create a lack of storing and loading from memory, which would lead to waiting for data coming from memory (memory bandwidth bottleneck). So in theory, model size would effect the performance non-linearly. In practice, we couldn't find a good way of predicting the number of inferences based on number of floating point operations and model size. 

One thing that greatly affects this process is the implementation of the model executor. We think that the tensorflow implementation of the model executor is not meant to load the whole model on the cpu. 

\iffalse
But as we understood, Tensorflow implementation is not executing models on model level, but on layer level (based on variable scopes). We think that is the case because when we look at a sufficiently small model's benchmarking results, we saw that the amount of context switches were equal to the number of variable scopes times the number of inferences.
\fi