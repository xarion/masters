% !TEX root = ../thesis.tex
\iffalse
Introduction
You can't write a good introduction until you know what the body of the paper says. Consider writing the introductory section(s) after you have completed the rest of the paper, rather than before.

Be sure to include a hook at the beginning of the introduction. This is a statement of something sufficiently interesting to motivate your reader to read the rest of the paper, it is an important/interesting scientific problem that your paper either solves or addresses. You should draw the reader in and make them want to read the rest of the paper.

The next paragraphs in the introduction should cite previous research in this area. It should cite those who had the idea or ideas first, and should also cite those who have done the most recent and relevant work. You should then go on to explain why more work was necessary (your work, of course.)
 
What else belongs in the introductory section(s) of your paper? 

    A statement of the goal of the paper: why the study was undertaken, or why the paper was written. Do not repeat the abstract. 
    Sufficient background information to allow the reader to understand the context and significance of the question you are trying to address. 
    Proper acknowledgement of the previous work on which you are building. Sufficient references such that a reader could, by going to the library, achieve a sophisticated understanding of the context and significance of the question.
    The introduction should be focused on the thesis question(s).  All cited work should be directly relevent to the goals of the thesis.  This is not a place to summarize everything you have ever read on a subject.
    Explain the scope of your work, what will and will not be included. 
    A verbal "road map" or verbal "table of contents" guiding the reader to what lies ahead. 
    Is it obvious where introductory material ("old stuff") ends and your contribution ("new stuff") begins? 

Remember that this is not a review paper. We are looking for original work and interpretation/analysis by you. Break up the introduction section into logical segments by using subheads. 
\fi

The state of the art in Image Processing has changed when graphics processing units (GPU) were used to train neural networks. GPUs contain many cores, they have very large data bandwidth and they are optimized for efficient matrix operations. In 2012, \cite{krizhevsky2012imagenet} won the ImageNet Large Scale Visual Recognition Competition (ILSVRC) classification task (\cite{deng2012image}). They used two GPUs to train an 8 layer convolutional neural network (CNN). Their model has improved the previous (top-5) classification accuracy record from $\sim 74\%$ to $\sim 84\%$. This caused a big trend shift in Computer Vision. 

As the years pass, GPUs got more and more powerful. In 2012, \cite{krizhevsky2012imagenet} used GPUs that had 3 GB memory. Today there are GPUs with up to 12 GB memory. The number of floating point operations per second (FLOPs) has also increased from $2.5$ tera FLOPs (TFLOPs) to $12$ TFLOPs. This gradual but steep change has allowed the use of more layers and more parameters. For example, \cite{Simonyan:2014aa} introduced a model called VGGNet. Their model used up to 19 layers and shown that adding more layers affects the accuracy. \cite{He:2015aa} introduced a new method called residual connections, that allowed the use of up to 200 layers. Building up on such models, in 2016 ILSVRC winning (top-5) classification accuracy is increased to $\sim 97\%$. 

In contrast, \cite{Szegedy:2014aa} have shown that having a good harmony within the network worked better than having more parameters. It has been supported by \cite{Canziani:2016aa}. They have shown the relation between number of parameters of a model and its top-1 classification accuracy in ILSVRC dataset. According to their report, 48 layer Inception-v3 (\cite{Szegedy_2016_CVPR}) provides better top-1 classification accuracy than 152 layer ResNet (\cite{He:2015aa}). They also show that Inception-v3 requires fewer number of floating point operations to compute results. Therefore, revealing that of providing more layers and parameters would not yield better results. 

ILSVRC is one of the most famous competitions in Image Processing. Every year, the winners of this competition are a driving the research on the field. But this competition is not considering the competitive value of limiting number of operations. If we look at the models of 2016 competitors, we see that they use ensembles of models\footnote{\url{http://image-net.org/challenges/LSVRC/2016/results\#team}}. These ensembles are far from being usable in real life because they require a great amount of operations per inference. Not mentioning the number of operations from the result is misleading for the AI community and the public. It creates an unreal expectation that these models are applicable in real life. In this thesis, we want to come up with a state of the art solution that requires a low number of floating point operations per inference. Therefore, bridging the gap between expectations and reality.

\section{Neural Networks}
In this chapter, we will try to describe neural networks briefly. To keep things simple, we only are concerned with feed forward neural networks and supervised learning. We will provide some terminology and give some examples. 

Neural networks are \textit{weighted graphs}. They consist of an ordered set of \textit{layers}, where every layer is a set of \textit{nodes}. The first layer of the neural network is called the \textit{input layer}, and the last one is called the \textit{output layer}. The layers in between are called \textit{hidden layers}. In our case, nodes belonging to one layer are connected to the nodes in the following and/or the previous layers. These connections are weighted edges, and they are mostly called as \textit{weights}. 

Given an input, neural networks nodes have \textit{outputs}, which are real numbers. The output of a node is calculated by applying a function ($\psi$) the outputs of the nodes belonging to previous layers . Preceding that, the output of the input layer is calculated using the input data (see Eq. \ref{eq:output_of_layers}).  By calculating the layer outputs consecutively we calculate the output of the output layer. This process is called \textit{inference}. We use the following notations to denote the concepts that we just explained.
\begin{equation}
\label{eq:variable_definitions}
\begin{split}
l_k & \text{: a column vector of nodes for layer $k$}\\
m_k & \text{: the number of nodes in $l_k$}\\
l_{k,i}  & \text{: node $i$ in $l_k$}\\
o_{k}  & \text{: the output vector representing the outputs of nodes in $l_{k}$}\\
o_{k,i}  & \text{: the output of $l_{k,i}$}\\
\mathbf{w}^{(k)}  & \text{: weight matrix connecting nodes in $l_{k-1}$ to nodes in $l_{k}$} \\
w^{(k)}_{i,j}  & \text{: the weight connecting nodes $l_{(k-1),i}$ and $l_{k,j}$} \\
\mathbf{b}^{(k)}  & \text{: the bias term for $l_{k}$} \\
\psi_k & \text{: function to determine $o_k$ given $o_{k-1}$}\\
\sigma & \text{: activation functions} \\
\mathbf{x} & \text{: all inputs of the dataset, consisting of $N$ data points} \\
\mathbf{y} & \text{: all outputs of the dataset} \\
\mathbf{\hat y} & \text{: approximation of the output}  \\
x_n & \text{: $n$th input data ($0 < n \leq N$)} \\
y_n & \text{: $n$th output data ($0 < n \leq N$)} \\
\hat y_n & \text{: approximation of $y_n$ given $x_n$ ($0 < n \leq N$)}\\
\text{FC} & \text{: stands for Fully Connected (e.g. $\psi^{(FC)}$)}
\end{split}
\end{equation}
Therefore the structure of a neural network is determined by the number of layers and the functions that determine the outputs of layers.
\begin{equation}
\label{eq:output_of_layers}
    o_k = 
\begin{cases}
    \psi(o_{k-1}), &\text{if } k\geq 1\\
    \mathbf{x},& k = 0\\
\end{cases}
\end{equation}

\subsection{Fully Connected Layers}
As the name suggests, for two consecutive layers to be \textit{fully connected}, all nodes in the previous layer must be connected to all nodes in the following layer. 

Let's assume two consecutive layers, $l_{k-1}$ and $l_{k}$, with shapes $m_{k-1} \times 1$ and $m_k \times 1$, respectively. For these layers to be fully connected, the weight matrix $\mathbf{w}^{(k)}$, should be of shape $m_{k-1} \times m_{k}$. Most fully connected layers also include a bias term ($m$) for every node $l_k$. In fully connected layers, $o_k$ would simply be calculated using layer function $\psi^{(FC)}$.
$$ \psi^{(FC)}_k(o_{k-1}) = o_{k-1}^T\mathbf{w}^{(k)} + \mathbf{b}^{(k)}$$
Therefore the computational complexity of $\psi^{(FC)}$ would become
$$\mathcal{O}(\psi^{(FC)}_k) = m_{k-1}m_{k}$$

\subsection{Activation Function and Nonlinearity}
By stacking fully connected layers, we can increase the depth of a neural network. By doing so we want to increase approximation quality of the neural network. However, the $\psi^{(FC)}$ we have defined is a linear function. Therefore if we stack multiple fully connected layers we would end up with a linear model. 

To achieve non-linearity, we apply \textit{activation functions} to the results of $\psi$. There are many activation functions (such as $tanh$ or $sigmoid$) but one very commonly used activation function is $ReLU$.  
\begin{equation}
\label{eq:relu_definition}
    ReLU(x) = 
\begin{cases}
    x, & \text{if }x \geq 0\\
    0 &  \text{otherwise }\\
\end{cases}
\end{equation}
Therefore we will redefine the fully connected $\psi^{(FC)}$ as;

$$ \psi^{(FC)}_k(o) = \sigma(o^T\mathbf{w}^{(k)} + \mathbf{b}^{(k)})$$

$\psi^{(FC)}$ is one of the most basic building blocks of any Neural Network. Stacking $K$ of them after the input, we can try to approximate an output given an input. To do that we will calculate the outputs of every layer, starting from the input. 
\begin{equation*}
\begin{split}
o_0 &= x_n\\
o_1 &= \psi_1^{(FC)}(o_0)\\ 
o_2 &= \psi_2^{(FC)}(o_1)\\
...&\\
o_K &= \psi_1^{(FC)}(o_{K-1})\\
\hat y &= o_K
\end{split}
\end{equation*}

\subsection{Loss}
To represent the quality of an approximation, we are going to use a loss (or cost) function. A good example would be the loss of a salesman. Assuming a customer who would pay at most \$10 for a given product, if the salesman sells this product for \$4, the salesman would face a loss of \$6 from his potential profit. Or if the salesman tries to sell this product for \$14, the customer will not purchase it and he will face a loss of \$10. In this example, the salesman would want to minimize the loss to earn as much as possible. 

A commonly used loss function is Root Mean Square Error (RMSE). Given an approximation ($\hat y$) and the corresponding output ($y$), RMSE can be calculated as,
\begin{equation*}
L = RMSE(\hat y, y) = \sqrt{\frac{\sum^N_{n=1} (\hat y_n - y_n)^2 }{N}}
\end{equation*}
There are two common properties of loss functions. First, loss is never negative. Second, if we compare two approximations, the one with a smaller loss is better.

If all our approximations are exactly the same as the output ($  \mathbf{\hat y} =  \mathbf{y} $), the total loss would be 0. 

\subsection{Stochastic Gradient Descent}
To provide better approximations, we will try to optimize the neural network parameters. One common way to optimize these parameters is to use Stochastic Gradient Descent (SGD). SGD is an iterative learning method that starts with some initial (random) parameters. Given $\theta \in \mathbf{w} \cup \mathbf{b}$ to be a parameter that we want to optimize. The learning rule updating theta would be;

$$ \theta = \theta- \eta \nabla_\theta{L(f(x), y)} $$

where $\eta$ is the learning rate, and $\nabla_\theta{L(f(x), y)}$ is the partial derivative of the loss in terms of given parameter, $\theta$. One iteration is completed when we update every parameter for given example(s). By performing many iterations, SGD aims to find a global minimum for the loss function, given data and initial parameters.

\subsection{Convolutional Layer}
So far we have seen the key elements we can use to create and train fully connected neural networks. To be able to apply neural networks to image inputs, we will use convolutional layers or convolution operation. Let's assume an image $i \in \mathcal{R}^{W \times H \times C}$. Convolution operation first creates a sliding window that goes through the image. We will call the contents of this sliding window with centered $(i,j)$ as a patch ($p_{(i,j)} \in \mathbb{R}^{K \times K \times C}$). Convolution operation sees every element of this patch as a node. By multiplying a weight matrix $\mathbf{w}^{(k)} \in \mathbb{R}^{K \times K \times C \times O}$ to this patch of nodes, it creates a set of output nodes $o_{(k+1),(i,j)} \in \mathbb{R}^{1 \times O}$. Those output nodes represent the features belonging to the pixel at point $(i,j)$. By performing this operation for every patch, we calculate the outputs of a convolutional layer. 

A convolutional layer with kernel size 3 would combine the data from all neighboring pixels while calculating the output information. Therefore it's outputs can represent shapes. By adding another convolutional layer, we would combine information of neighboring shapes, therefore represent more complex shape information. By adding more convolutional layers, we can represent more and more complex shapes, or objects, or stuffs. Such networks are called convolutional neural networks (CNN). 

The complexity of the matrix multiplication on a patch is $\mathcal(KKIO)$. Complexity of applying this operation to every patch is $\mathcal{O}(WHK^2CO)$. 

\todoin{shall we provide more details?}

\iffalse
The convolutional layer creates $K \times K \times C$ patches from a given input. Then it multiplies this patch with a weight matrix of size $K \times K \times C \times O$ where $O$ is the number of output channels. Let's assume that we have a patch from a layer $k$ at point $(I, J)$; 

\todoin{$p(o_k, I,J, K)$ could also be something like $p^{(I,K)}_{k,(i.j)}$ also maybe check bishop or something to see a better definition. maybe include padding and strides as well. tell what a kernel is, define output/input channels better in the definition of a patch.}
\begin{equation}
\label{eq:convolution_patches}
\begin{split}
    p^{(I,K)}_{k,(i.j)} = &
\begin{cases}
    o_{k,(I-K/2+i, J-K/2+j)}, & \text{if } (0,0) < (I-K/2+i, J-K/2+j) \leq (W,H)\\
    0 &  \text{ otherwise }\\
\end{cases}\\ 
&\text{where } 0 \leq i < I \text{ and } 0 \leq j < J
\end{split}
\end{equation}

$$ o_{k, (I,J)} = \psi_k^{(Conv)}() =  p\mathbf{w_k}$$
$$ o_{k+1,(i,j)} = \psi^{(Conv)}_k(o_{k-1}) $$
\fi


\subsection{Pooling}
Pooling is a way of reducing the dimensionality of an image. Depending on the task, one may use from different pooling methods. Pooling methods create patches of size $k$ that are distant from each other by $s$ pixels in horizontally and vertically. $s$ is called the stride of pooling and $k$ is referred as the kernel size. For example, pooling an image with $W \times H \times C$ dimensions with strides of 2 would result with an image of dimensions $W/2 \times H/2 \times C$. Pooling methods do not change the amount of channels the input has. But they choose values for the channels based on the pooling method.

Average pooling averages the values within the patch per channel. Max pooling takes the maximum value in a channel within the patch. Global Average Pooling takes the whole image as 1 patch and applies method max/average pooling on that 1 patch. Therefore returns only a matrix of dimensions $1 \times 1 \times C$.

\section{Efficient Operations}
In this section we are going to look at some alternatives to convol

ion operation with better complexity.

\subsection{1D Convolutions}
Convolution kernels could be decomposed into smaller operations. For example using SVD, we can decompose the convolution kernel into the multiplication for two smaller kernels, and by applying these kernels consequently, we can approximate the convolution operation with fewer floating point operations. Instead of doing separating learned convolutions, we are going to use the method introduced in \cite{alvarez2016decomposeme}. This method forces the separability of convolution operation as a hard constraint. 

\subsection{Separable Convolutions}
As the name suggests, these operations separate the standard convolution operation into two parts. These parts are called Depthwise convolutions and Pointwise convolutions. Depthwise convolution applies a given number of filters on every input channel, one by one therefore results with output channels equal to input channel times number of filters. Pointwise convolution correlates these output channels with each other, or in other words mixes them, by applying a matrix multiplication. For example, let's assume we have an input with 3 channels, applying a Depthwise convolution with 4 filters to that, we would get 12 output channels. Then applying a Pointwise convolution to that, we correlate these 12 output channels to create new output channels. 

To describe the complexity of this operation, let's assume that we have a separable convolution with kernel size $N$, input channels $K$, depthwise filters $I$ and output channels $L$. First operation will be applying $L$ filters with size $N \times N$ to $K$ input channels, one by one. The number of operations we need for this operation is, $IKN^2$. Second operation will be multiplying $1 \times IK$ output with $IK \times L$ correlation matrix, requiring $IKL$ floating point operations. In total we need $IK(N^2+L)$ operations. 


Normally convolution operations are defined 2 dimensional ($2D$). That is, kernel sizes are ($2D$). For example convolutions we have used in Section \ref{sec:activation-based-pruning-convolution} are $2D$, their kernel sizes are $3 \times 3$. With this method, we are aiming to construct a $N \times N$ convolution operation as a combination two convolution operations. The first convolution has kernel size $1 \times N$ and the following convolution has $N \times 1$, or vice versa. Looking back at the experiment we did in  Section \ref{sec:activation-based-pruning-convolution}, instead of applying one $3 \times 3$ convolution, we are talking about applying one $1 \times 3$ convolution followed by a $3 \times 1$ convolution. 

The amount of speed up can be approximated using the number of floating point operations. A convolution operation can be seen as a matrix multiplication applied to consequent subsections of an image. To visualize this, let's assume that we have a convolution operation with kernel size $N$, input channels $K$ and output channels $L$. Assuming we are applying that operation to a $N \times N$ image patch with $K$ channels, our operation can be simplified as multiplying this $N*N*K$ vector with $N*N*K \times L$ matrix. The number of floating point operations required to execute this operation are $N^2KL$. When composed as two $1D$ compositions, this operation will be represented with two $1D$ convolutions. First convolution is multiplying vector $N*K$ with matrix $N*K \times P$ and the second convolution is multiplying vector $N*P$ with matrix $N*P \times L$. As you can see we have introduced the variable $P$ as the intermediate output of the first convolution. Number of floating point operations required for these operations are, $NLP$ and $NPK$. And summing them up, we can say we need $NP(L + K)$ operations in total. To see if this decomposition is reducing the number of operations, we could basically try to define a $P$ satisfying $\frac{NP(L+K)}{N^2KL} < 1$. Therefore,
\begin{equation*}
1 \leq P < \frac{NKL}{L+K}
\end{equation*}

To show the validity of this method, we have conducted two experiments. 

\section{Pruning}
Pruning aims to reduce the number of operations by deleting the parameters that has low or no impact in the result. Studies show that applying this method in an ANN is effective in reducing the model complexity, improving generalization, and they are effective in reducing the required training cycles. In our experiments we will try to reproduce these effects.
To visualize these methods, and help with the explanation later, let's think of two fully connected layers, $\mathbf{l_1}$ and $\mathbf{l_2}$. $\mathbf{l_1}$ is the input of this operation and it consists of $N$ values, $\mathbf{l_1}=(l_{11}, l_{12}, ..., l_{1N})$. $\mathbf{l_2}$ is the output of this operation consists of $M$ values, $\mathbf{l_2}=(l_{21}, l_{22}, ..., l_{2M})$. Between these two layers, there is a weight matrix $W$ with size $N \times M$. The operation, that we want to optimize is, $\mathbf{l_2} = \mathbf{l_1}W$. To do so, we will look at 2 cases of pruning. One will be focusing on pruning individual weights, and the other will be focusing on removing unimportant rows and columns from $\mathbf{l_2}$, $\mathbf{l_1}$ and $W$. 
\todoin{maybe explain in more detail and give examples of pruning algorithms here. (e.g. Optimal Brain Damage, Second order derivatives for network pruning: Optimal Brain Surgeon, Optimal Brain Surgeon and general network pruning, SEE Pruning Algorithms-a survey from R. Reed)}

\subsection{Pruning Weights}
This subcategory of pruning algorithms try to optimize the number of floating point operations by removing some individual values from $W$. Theoretically, it could benefit the computational complexity to remove individual scalars from W, by not performing operations related to those weights. But practically, in Tensorflow, matrix multiplication on dense matrices uses all of the values of it's inputs. In contrary, sparse matrix multiplication takes a sparse matrix and a dense matrix as inputs and outputs a dense matrix. To implement this method, we could convert W to a sparse tensor after pruning the weights. But, Tensorflow documentations explicitly state;
\begin{displayquote}
\begin{itemize}
\item Will the SparseTensor $A$ fit in memory if densified?
\item Is the column count of the product large ($>> 1$)?
\item Is the density of $A$ larger than approximately $15\%$?
\end{itemize}
"If the answer to several of these questions is yes, consider converting the SparseTensor to a dense one."
\end{displayquote}
In our terms, SparseTensor $A$ is corresponding to the pruned version of $W$. Since $W$ was already dense before, we can assume that the answer to the first question is yes. The column count of our product is $M$ which is much larger than $1$ in some cases. Also we don't know anything about the density of pruned version of $W$. Looking at these facts, we are assuming that implementing this operation will be problematic. Instead of delving deeper into these problems to evaluate this method, we will move on to other methods.
\todoin{add figure to show what happens when we prune}
\subsection{Activation Based Pruning - Fully Connected Layers} \label{sec:activation-based-pruning-convolution}
Activation based pruning, works by looking at individual values in layers, and prunes the layer and corresponding weight row/columns completely. To visualize this, we will assume that the fully connected layers we have defined are, trained to some extent, and activated using ReLU activations. With this definition, if we apply our dataset and count the number of activations in $\mathbf{l_1}$ and $\mathbf{l_2}$, we may realize that there are some neurons that are not being activated at all. By removing these neurons from the layers, we can reduce the number of operations. This removal operation is done by removing neurons based on their activations. 
\todoin{add figure to show what happens when we prune}

\subsection{Activation Based Pruning - Convolution and Deconvolutions}
\todoin{Put references for conv and deconv operations. }
In theory, convolution operation is a matrix multiplication applied on a sliding window. Thus, counting the output feature activations of a convolution operation, we can apply activation based pruning. 

\subsection{Second Order Derivatives (Fischer Information Matrix)}

\section{Factorization}
Using Factorization methods, we can decompose a matrix into smaller different matrices. Some factorization methods can be used to reduce the dimensionality of these smaller matrices while approximating the original matrix. This has interesting uses with Neural Networks. Assume that we have a matrix multiplication operation. We are multiplying a random input $\mathbf{X}$ with a fixed weight matrix $\mathbf{W}$. Let's say $\mathbf{X}$ is $K \times N$ and $\mathbf{W}$ is $N \times M$. The matrix multiplication of these two matrices has the complexity of $KNM$. If we can successfully decompose $\mathbf{W}$ to the composition of two matrices $\mathbf{O}$ and $\mathbf{P}$ with dimensions $N \times L$ and $L \times M$, respectively, we can rewrite our matrix multiplication operation as; $\mathbf{X}\mathbf{W} \approx \mathbf{X}\mathbf{O}\mathbf{P}$. The new complexity of this operation would be $KNL + NLM = NL(K+M)$. If we can find a decomposition where $L$ is sufficiently small that successfully approximates the matrix $\mathbf{W}$ and satisfies $NL(K+M) < KNM$, we can reduce the complexity of this matrix multiplication.
\todoin{draw some figures explaining how this happens}
\todoin{Luc says: If XW is not exactly equal XOP, how will you deal with this?}
\subsection{SVD}
Using SVD we can make this approximation. SVD decomposes the a matrix into 3 parts. 
\todoin{Talk what SVD does, what are the decomposed matrices, what are the singular values and how they are relevant to the approximation of $\mathbf{W}$. Show your experiments and results from when you ran the experiments.} 


\subsection{Weight Sharing}
Weight sharing assumes we have a limited set of weights, and when we are representing values, instead of representing the value itself, we represent the indices to weights. This operation is used in some papers successfully to reduce model size considerably. To be able to implement such a representation, we can use 2 potential functions of Tensorflow. One is \texttt{tf.gather} which selects the given indices from a given matrix. The other is \texttt{tf.embedding\_lookup} which works with a bucket of values and returns the given keys/indices from the given bucket. However, both methods accept the indices as a matrix of 32-bit or 64-bit integers. Therefore storing indices instead of weights would not yield with any improvements in the model size. Without an implementation of these methods using low-bit integers, it is not possible to exploit their usefulness.
\todoin{give some examples here, you're saying some papers.}

\subsection{Other Factorization Methods}


\section{Quantization}

\subsection{8-bit Quantization}
\subsection{n-bit Quantization}

\section{Efficient Structures}
Some structures help neural networks represent more information using less parameters.
\todoin{talk more about why some structures are more efficient, how they help with training speed, how they reduce the number parameters or number of floating point operations even while increasing the accuracy.}

\subsection{Inception Blocks}
\todoin{do the introduction to \cite{Szegedy:2014aa} and how it is improved using \cite{Szegedy_2016_CVPR}}
\subsection{Bottleneck Blocks}
\cite{He:2015aa} introduced residual connections with bottleneck blocks. To optimize the performance of their network, they have introduced the bottleneck blocks. Bottleneck blocks contain 3 convolutions. First is a $ 1 \times 1$ convolution that scales down the input channels to half. Output is applied to a $3 \times 3$ convolution which doesn't change the number of channels, and following that with a $1 \times 1$ convolution to quadruple the number of input channels. As an example, we can look at the conv3\_x block of 50-layer network described in Table \ref{tab:bottleneck-comparison}.
\todoin{try to reason why bottleneck blocks work. Luc says: what is the reasoning of this? why would one want to do this?}

\cite{He:2015aa} compared the performance of various network configurations on ImageNet validation dataset. From these comparisons, we have selected the 34-layer network and the 54-layer network. The 34-layer network is consisting of pairs of $3 \times 3$ blocks. The 50-layer network is consisting of bottleneck blocks. In table \ref{tab:bottleneck-comparison} we have compared these networks by their structure, required number of FLOPs, and their top-1 and top-5 errors on this dataset. As we can see in the FLOPs, the networks have about $5\%$ of difference in number of floating point operations. As \cite{He:2015aa} reports, this small increase in parameters is effecting accuracy of the model considerably. 

\begin{table}[]
\centering
\begin{tabular}{ | c | c | c | c | }
\hline
layer name			& output size 					& 34-layer																& 50-layer																			\\ \hline
input image			& $224 \times 224$				& \multicolumn{2}{c|}{}																																	\\ \hline
conv1				& $112 \times112$				& \multicolumn{2}{c|}{$ 7 \times 7$, $64$, stride $2$}																												\\ \hline
\multirow{2}{*}{conv2\_x}	& \multirow{2}{*}{$56 \times 56$} 	& \multicolumn{2}{c|}{$3 \times 3$ max pool, stride $2$}																											\\ \cline{3-4} 
					&							& $\begin{bmatrix} 3 \times 3, &   64 \\ 3 \times 3, &   64 \end{bmatrix} \times 3 $		& $\begin{bmatrix}1 \times 1, & 64 \\ 3 \times 3, & 64 \\ 1 \times 1, & 256 \end{bmatrix}^{} \times 3 $ 		\\ \hline
conv3\_x				& $28 \times 28$				& $\begin{bmatrix} 3 \times 3, & 128 \\ 3 \times 3, & 128 \end{bmatrix} \times 3 $		& $\begin{bmatrix}1 \times 1, & 128 \\ 3 \times 3, & 128 \\ 1 \times 1, & 512 \end{bmatrix} \times 3$		\\ \hline
conv4\_x				& $14 \times 14$				& $\begin{bmatrix} 3 \times 3, & 256 \\ 3 \times 3, & 256 \end{bmatrix} \times 3 $		& $\begin{bmatrix}1 \times 1, & 256 \\ 3 \times 3, & 256 \\ 1 \times 1, & 1024 \end{bmatrix} \times 3$		\\ \hline
conv5\_x				& $  7 \times   7$				& $\begin{bmatrix} 3 \times 3, & 512 \\ 3 \times 3, & 512 \end{bmatrix} \times 3 $		& $\begin{bmatrix}1 \times 1, & 512 \\ 3 \times 3, & 512 \\ 1 \times 1, & 2048 \end{bmatrix} \times 3$		\\ \hline
					& $  1 \times   1$				&\multicolumn{2}{c|}{average pool, 1000-d fc, softmax}																											\\ \hline
\multicolumn{2}{| c |}{FLOPs}							& $3.6 \times 10^9$														& $3.8 \times 10^9$																	\\ \hline
\multicolumn{2}{| c |}{top-1 error ($\%$)}						& $21.53$																& $20.74$																			\\ \hline
\multicolumn{2}{| c |}{top-5 error ($\%$)}						& $5.60$																& $5.25$																			\\ \hline
\multicolumn{2}{| c |}{top-1 error \small{($\%$, \textbf{10-crop} testing)}}						& $24.19$																& $22.85$																			\\ \hline
\multicolumn{2}{| c |}{top-5 error \small{($\%$, \textbf{10-crop} testing)}}						& $7.40$																& $6.71$																			\\ \hline
\end{tabular}
\caption{Comparison of bottleneck blocks (50-layer) with stacked $ 3 \times 3$ layers (34-layer). }
\label{tab:bottleneck-comparison}
\end{table}

But the main contribution of \cite{He:2015aa} is not the bottleneck architecture, but Residual Connections that we will see in another section. 


\section{Improving Network Efficiency}
\subsection{Residual Connections}
\cite{He:2015aa} is also introducing a method called Residual Connections.
\todoin{go into details of how this works using information given in \cite{He:2015aa} and \cite{DBLP:journals/corr/SzegedyIV16}}
\subsection{Batch Normalization}
\begin{equation}
\begin{split}
V \approx& HW \\
IV \approx& IHW\\
I \in& \mathbb{R}^{l \times m}
\end{split}
\end{equation}


\section{Datasets}


























